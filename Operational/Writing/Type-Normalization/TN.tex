\documentclass[authoryear, acmsmall, screen, review, nonacm]{acmart}
\overfullrule=1mm
% \usepackage[margin=1.5in]{geometry}

\newcommand\Root{/home/alex/Dropbox/HubersPhD/bibliography}
\usepackage{
  enumitem,
  amsmath,
%  amsthm,
%  amssymb,
  xspace,
  mathtools,
  multicol,
  tikz-cd,
  parskip,
  amsfonts,
  mathrsfs,
  stmaryrd,
  tikz,
  float,
  titlecaps,
  soul,
  upgreek,
  subfiles,
  caption,
  graphicx,
  array,
  booktabs,
  subcaption,
  tikz,
  tikz-cd,
  centernot,
  ifthen,
  thmtools,
  fancyvrb,
  \Root/sty/mathwidth,
  \Root/sty/chronology,
  %\Root/sty/sectsty,
  \Root/sty/infer,
}

% -----------------------------------------------------------------------------
% Conditionals
\newif\ifstretch
\newif\ifdarkmode
\newif\ifcomments
\newif\iffastercode

\fastercodefalse
\commentstrue
\darkmodefalse
\stretchfalse

\ifdarkmode \pagecolor[rgb]{0,0,0} \color[rgb]{1,1,1} \fi

% -----------------------------------------------------------------------------
% My sty's
\usepackage{TN}

% -----------------------------------------------------------------------------
% ulem
\usepackage[normalem]{ulem}

% -----------------------------------------------------------------------------
% natbib
%\usepackage[square,numbers]{natbib}

% -----------------------------------------------------------------------------
% tikz
\usetikzlibrary{positioning}
\tikzset{main node/.style={circle,fill=white!20,draw,minimum size=1cm,inner sep=2pt}}

% -----------------------------------------------------------------------------
% Times New Roman, as per ACM.
%\usepackage{mathptmx}
%\DeclareMathAlphabet{\mathcal}{OMS}{cmsy}{m}{n}


% -----------------------------------------------------------------------------
% liststep and parindent
\setlist{nosep}
\setlist[enumerate]{nosep}
\setlength{\parindent}{10pt}


% -----------------------------------------------------------------------------
% Set counter to 0 (for abstract)
\setcounter{section}{0}


% -----------------------------------------------------------------------------
% Blank page
\newcommand\blankpage{% comando pagina vuota
    \clearpage
    \null
    \thispagestyle{empty}%
    \addtocounter{page}{-1}%
    \clearpage
}
% -----------------------------------------------------------------------------
% cleverref

\usepackage[capitalize,noabbrev]{cleveref}
\crefformat{section}{(\S#2#1#3)}
\crefformat{subsection}{(\S#2#1#3)}
\crefformat{subsubsection}{(\S#2#1#3)}
\crefmultiformat{section}{(\S\S#2#1#3}{ and~#2#1#3}{, #2#1#3}{, and~#2#1#3)}
\crefmultiformat{subsection}{(\S\S#2#1#3}{ and~#2#1#3}{, #2#1#3}{, and~#2#1#3)}
\crefmultiformat{subsubsection}{(\S\S#2#1#3}{ and~#2#1#3}{, #2#1#3}{, and~#2#1#3)}

% -----------------------------------------------------------------------------
% math envs


% \ifdefined\theorem \else
%   \newtheorem{theorem}{Theorem}
% \fi
% \newtheorem{assumption}[theorem]{Assumption}
% \newtheorem{prop}[theorem]{Proposition}
% \newtheorem{claim}[theorem]{Claim}
% \theoremstyle{definition}
% \newtheorem{defn}[theorem]{Definition}
% \ifdefined\example \else
%   \newtheorem{example}[theorem]{Example}
% \fi
% \theoremstyle{remark}
% \newtheorem*{remark}{Remark}

% \newcommand{\indextheorem}[1]{\index[theorems]{#1}} 
% \newcommand{\indexdefinition}[1]{\index[definitions]{#1}}

% \AtEndPreamble{%
%   \theoremstyle{acmplain}
%   %\newtheorem{theorem}{Theorem}
%   \newtheorem*{theorem*}{Theorem}}

% \newenvironment{AgdaDefs}%
%   {\begin{smalle}\[\begin{array}{l@{\;}c@{\;}l} }%
%   {\end{array}\]\end{smalle}\ignorespacesafterend}

% \newenvironment{AgdaDefsData}%
%   {\begin{smalle}\[\begin{array}{l@{\;}l@{\;}c@{\;}l} }%
%   {\end{array}\]\end{smalle}\ignorespacesafterend}
% %%% Local Variables: 
% %%% TeX-command-extra-options: "-shell-escape"
% %%% End: 
\usepackage{agda}
\usepackage[utf8]{inputenc}
\usepackage[T1]{fontenc}

% ACM garbage
\setcopyright{none}
\citestyle{acmauthoryear}
\settopmatter{printacmref=false, printfolios=true}
\renewcommand{\footnotetextcopyrightpermission}{} 
% Redefine the \acmDOI command to do nothing 
\pagestyle{empty}
\fancyfoot{}

%\usepackage[utf8]{inputenc}
% \numberwithin{equation}{section}
% \numberwithin{theorem}{section}

\title{Type Normalization in \Rome}
\author{Alex Hubers}
\orcid{0000-0002-6237-3326}
\affiliation{
  \department{Department of Computer Science}
  \institution{The University of Iowa}
  \streetaddress{14 MacLean Hall}
  \city{Iowa City}
  \state{Iowa}
  \country{USA}}
\email{alexander-hubers@uiowa.edu}

\usepackage{newunicodechar}
\newunicodechar{∋}{$\ni$}
\newunicodechar{ε}{$\epsilon$}
\newunicodechar{·}{$\cdot$}
\newunicodechar{⊢}{$\vdash$}
\newunicodechar{⋆}{${}^\star$}
\newunicodechar{Π}{$\Pi$}
\newunicodechar{⇒}{$\Rightarrow$}
\newunicodechar{ƛ}{$\lambdabar$}
\newunicodechar{∅}{$\emptyset$}
\newunicodechar{∀}{$\forall$}
\newunicodechar{ϕ}{$\Phi$}
\newunicodechar{ψ}{$\Psi$}
\newunicodechar{ρ}{$\rho$}
\newunicodechar{α}{$\alpha$}
\newunicodechar{β}{$\beta$}
\newunicodechar{μ}{$\mu$}
\newunicodechar{σ}{$\sigma$}
\newunicodechar{≡}{$\equiv$}
\newunicodechar{Γ}{$\Gamma$}
\newunicodechar{∥}{$\parallel$}
\newunicodechar{Λ}{$\Lambda$}
\newunicodechar{₂}{$_2$}
\newunicodechar{θ}{$\theta$}
\newunicodechar{Θ}{$\Theta$}
\newunicodechar{∘}{$\circ$}
\newunicodechar{Δ}{$\Delta$}
\newunicodechar{★}{$\star$}
\newunicodechar{λ}{$\lambda$}
\newunicodechar{⊧}{$\models$}
\newunicodechar{⊎}{$\uplus$}
\newunicodechar{η}{$\eta$}
\newunicodechar{⊥}{$\bot$}
\newunicodechar{Σ}{$\Sigma$}
\newunicodechar{ξ}{$\xi$}
\newunicodechar{₁}{$_1$}
\newunicodechar{₃}{$_3$}
\newunicodechar{ℕ}{$\mathbb{N}$}
\newunicodechar{ᶜ}{${}^c$}
\newunicodechar{Φ}{$\Phi$}
\newunicodechar{Ψ}{$\Psi$}
\newunicodechar{⊤}{$\top$}
\newunicodechar{κ}{$\kappa$}
\newunicodechar{τ}{$\tau$}

\newunicodechar{→}{$\rightarrow$}
\newunicodechar{×}{$\times$}

\begin{document}

\maketitle

\section{Introduction}
We describe the normalization-by-evaluation (NBE) of types in \Rome. Types are normalized modulo $\beta$- and $\eta$-equivalence---that is, to $\beta\eta$-long forms. Because the type system of \Rome is a strict extension of System \FO, type level computation for arrow kinds is isomorphic to reduction of arrow types in the STLC. Novel to this report are the reductions of $\Pi$, $\Sigma$, and label bound terms. 

\section{Syntax of kinds}
Our formalization of \Rome types is \emph{intrinsic}, meaning we define the syntax of \emph{typing} and \emph{kinding judgments}, foregoing any description of untyped syntax. The syntax of types is indexed by kinding environments and kinds, defined below.

\begin{code}%
\>[0]\AgdaKeyword{data}\AgdaSpace{}%
\AgdaDatatype{Kind}\AgdaSpace{}%
\AgdaSymbol{:}\AgdaSpace{}%
\AgdaPrimitive{Set}\AgdaSpace{}%
\AgdaKeyword{where}\<%
\\
\>[0][@{}l@{\AgdaIndent{0}}]%
\>[2]\AgdaInductiveConstructor{★}%
\>[8]\AgdaSymbol{:}\AgdaSpace{}%
\AgdaDatatype{Kind}\<%
\\
%
\>[2]\AgdaInductiveConstructor{L}%
\>[8]\AgdaSymbol{:}\AgdaSpace{}%
\AgdaDatatype{Kind}\<%
\\
%
\>[2]\AgdaOperator{\AgdaInductiveConstructor{\AgdaUnderscore{}`→\AgdaUnderscore{}}}\AgdaSpace{}%
\AgdaSymbol{:}\AgdaSpace{}%
\AgdaDatatype{Kind}\AgdaSpace{}%
\AgdaSymbol{→}\AgdaSpace{}%
\AgdaDatatype{Kind}\AgdaSpace{}%
\AgdaSymbol{→}\AgdaSpace{}%
\AgdaDatatype{Kind}\<%
\\
%
\>[2]\AgdaOperator{\AgdaInductiveConstructor{R[\AgdaUnderscore{}]}}%
\>[8]\AgdaSymbol{:}\AgdaSpace{}%
\AgdaDatatype{Kind}\AgdaSpace{}%
\AgdaSymbol{→}\AgdaSpace{}%
\AgdaDatatype{Kind}\<%
\\
%
\\[\AgdaEmptyExtraSkip]%
\>[0]\AgdaKeyword{infixr}\AgdaSpace{}%
\AgdaNumber{5}\AgdaSpace{}%
\AgdaOperator{\AgdaInductiveConstructor{\AgdaUnderscore{}`→\AgdaUnderscore{}}}\<%
\end{code}

The kind system of \Rome defines $\star$ as the type of types; $L$ as the type of labels; $(\to)$ as the type of type operators; and $R[\kappa]$ as the type of \emph{rows} containing types at kind $\kappa$. As shorthand, we write $R^{n}[\kappa]$ to denote $n$ repeated applications of $R$ to the type $\kappa$--e.g., $R^3[\kappa]$ is shorthand for $R[ R[ R[ \kappa ]]]$.

The syntax of kinding environments is given below. Kinding environments are isomorphic to lists of kinds.

\begin{code}%
\>[0]\AgdaKeyword{data}\AgdaSpace{}%
\AgdaDatatype{KEnv}\AgdaSpace{}%
\AgdaSymbol{:}\AgdaSpace{}%
\AgdaPrimitive{Set}\AgdaSpace{}%
\AgdaKeyword{where}\<%
\\
\>[0][@{}l@{\AgdaIndent{0}}]%
\>[2]\AgdaInductiveConstructor{ε}\AgdaSpace{}%
\AgdaSymbol{:}\AgdaSpace{}%
\AgdaDatatype{KEnv}\<%
\\
%
\>[2]\AgdaOperator{\AgdaInductiveConstructor{\AgdaUnderscore{},,\AgdaUnderscore{}}}\AgdaSpace{}%
\AgdaSymbol{:}\AgdaSpace{}%
\AgdaDatatype{KEnv}\AgdaSpace{}%
\AgdaSymbol{→}\AgdaSpace{}%
\AgdaDatatype{Kind}\AgdaSpace{}%
\AgdaSymbol{→}\AgdaSpace{}%
\AgdaDatatype{KEnv}\<%
\end{code}

Let the metavariables $\Delta$ and $\kappa$ range over kinding environments and kinds, respectively. Correspondingly, we define \emph{generalized variables} in Agda at these names. The syntax of intrinsically well-scoped De-Bruijn-indexed variables is given below.

\begin{code}%
\>[0]\AgdaKeyword{private}\<%
\\
\>[0][@{}l@{\AgdaIndent{0}}]%
\>[2]\AgdaKeyword{variable}\<%
\\
\>[2][@{}l@{\AgdaIndent{0}}]%
\>[4]\AgdaGeneralizable{Δ}\AgdaSpace{}%
\AgdaGeneralizable{Δ₁}\AgdaSpace{}%
\AgdaGeneralizable{Δ₂}\AgdaSpace{}%
\AgdaGeneralizable{Δ₃}\AgdaSpace{}%
\AgdaSymbol{:}\AgdaSpace{}%
\AgdaDatatype{KEnv}\<%
\\
%
\>[4]\AgdaGeneralizable{κ}\AgdaSpace{}%
\AgdaGeneralizable{κ₁}\AgdaSpace{}%
\AgdaGeneralizable{κ₂}\AgdaSpace{}%
\AgdaSymbol{:}\AgdaSpace{}%
\AgdaDatatype{Kind}\<%
\\
%
\\[\AgdaEmptyExtraSkip]%
\>[0]\AgdaKeyword{data}\AgdaSpace{}%
\AgdaDatatype{KVar}\AgdaSpace{}%
\AgdaSymbol{:}\AgdaSpace{}%
\AgdaDatatype{KEnv}\AgdaSpace{}%
\AgdaSymbol{→}\AgdaSpace{}%
\AgdaDatatype{Kind}\AgdaSpace{}%
\AgdaSymbol{→}\AgdaSpace{}%
\AgdaPrimitive{Set}\AgdaSpace{}%
\AgdaKeyword{where}\<%
\\
\>[0][@{}l@{\AgdaIndent{0}}]%
\>[2]\AgdaInductiveConstructor{Z}\AgdaSpace{}%
\AgdaSymbol{:}\AgdaSpace{}%
\AgdaDatatype{KVar}\AgdaSpace{}%
\AgdaSymbol{(}\AgdaGeneralizable{Δ}\AgdaSpace{}%
\AgdaOperator{\AgdaInductiveConstructor{,,}}\AgdaSpace{}%
\AgdaGeneralizable{κ}\AgdaSymbol{)}\AgdaSpace{}%
\AgdaGeneralizable{κ}\<%
\\
%
\>[2]\AgdaInductiveConstructor{S}\AgdaSpace{}%
\AgdaSymbol{:}\AgdaSpace{}%
\AgdaDatatype{KVar}\AgdaSpace{}%
\AgdaGeneralizable{Δ}\AgdaSpace{}%
\AgdaGeneralizable{κ₁}\AgdaSpace{}%
\AgdaSymbol{→}\AgdaSpace{}%
\AgdaDatatype{KVar}\AgdaSpace{}%
\AgdaSymbol{(}\AgdaGeneralizable{Δ}\AgdaSpace{}%
\AgdaOperator{\AgdaInductiveConstructor{,,}}\AgdaSpace{}%
\AgdaGeneralizable{κ₂}\AgdaSymbol{)}\AgdaSpace{}%
\AgdaGeneralizable{κ₁}\<%
\end{code}

The kind variable $x$ is indexed by kinding environment $\Delta$ and kind $\kappa$ to specify that $x$ has kind $\kappa$ in kinding environment $\Delta$.



\section{Syntax of types}

\Rome is a qualified type system with predicates of the form $\rho_1 \lesssim \rho_2$ and $\rho_1 \cdot \rho_2 \sim \rho_3$ for row-kinded types $\rho_1$, $\rho_2$, and $\rho_3$. Because predicates occur in types and types occur in predicates, the syntax of well-kinded types and well-kinded predicates are mutually recursive. The syntax for each is given below; we describe (in this order) the syntactic components belonging to the STLC, System \FO, qualified types, and system \RO.

\begin{code}%
\>[0]\AgdaKeyword{data}\AgdaSpace{}%
\AgdaDatatype{Pred}\AgdaSpace{}%
\AgdaSymbol{(}\AgdaBound{Δ}\AgdaSpace{}%
\AgdaSymbol{:}\AgdaSpace{}%
\AgdaDatatype{KEnv}\AgdaSymbol{)}\AgdaSpace{}%
\AgdaSymbol{:}\AgdaSpace{}%
\AgdaDatatype{Kind}\AgdaSpace{}%
\AgdaSymbol{→}\AgdaSpace{}%
\AgdaPrimitive{Set}\<%
\\
\>[0]\AgdaKeyword{data}\AgdaSpace{}%
\AgdaDatatype{Type}\AgdaSpace{}%
\AgdaBound{Δ}\AgdaSpace{}%
\AgdaSymbol{:}\AgdaSpace{}%
\AgdaDatatype{Kind}\AgdaSpace{}%
\AgdaSymbol{→}\AgdaSpace{}%
\AgdaPrimitive{Set}\<%
\\
\>[0]\AgdaKeyword{data}\AgdaSpace{}%
\AgdaDatatype{Type}\AgdaSpace{}%
\AgdaBound{Δ}\AgdaSpace{}%
\AgdaKeyword{where}\<%
\\
%
\\[\AgdaEmptyExtraSkip]%
\>[0][@{}l@{\AgdaIndent{0}}]%
\>[2]\AgdaInductiveConstructor{`}%
\>[79I]\AgdaSymbol{:}\<%
\\
\>[79I][@{}l@{\AgdaIndent{0}}]%
\>[6]\AgdaSymbol{(}\AgdaBound{α}\AgdaSpace{}%
\AgdaSymbol{:}\AgdaSpace{}%
\AgdaDatatype{KVar}\AgdaSpace{}%
\AgdaBound{Δ}\AgdaSpace{}%
\AgdaGeneralizable{κ}\AgdaSymbol{)}\AgdaSpace{}%
\AgdaSymbol{→}\<%
\\
%
\>[6]\AgdaDatatype{Type}\AgdaSpace{}%
\AgdaBound{Δ}\AgdaSpace{}%
\AgdaGeneralizable{κ}\<%
\\
%
\\[\AgdaEmptyExtraSkip]%
%
\>[2]\AgdaInductiveConstructor{`λ}%
\>[87I]\AgdaSymbol{:}\<%
\\
\>[0]\<%
\\
\>[87I][@{}l@{\AgdaIndent{0}}]%
\>[6]\AgdaSymbol{(}\AgdaBound{τ}\AgdaSpace{}%
\AgdaSymbol{:}\AgdaSpace{}%
\AgdaDatatype{Type}\AgdaSpace{}%
\AgdaSymbol{(}\AgdaBound{Δ}\AgdaSpace{}%
\AgdaOperator{\AgdaInductiveConstructor{,,}}\AgdaSpace{}%
\AgdaGeneralizable{κ₁}\AgdaSymbol{)}\AgdaSpace{}%
\AgdaGeneralizable{κ₂}\AgdaSymbol{)}\AgdaSpace{}%
\AgdaSymbol{→}\<%
\\
%
\>[6]\AgdaDatatype{Type}\AgdaSpace{}%
\AgdaBound{Δ}\AgdaSpace{}%
\AgdaSymbol{(}\AgdaGeneralizable{κ₁}\AgdaSpace{}%
\AgdaOperator{\AgdaInductiveConstructor{`→}}\AgdaSpace{}%
\AgdaGeneralizable{κ₂}\AgdaSymbol{)}\<%
\\
%
\\[\AgdaEmptyExtraSkip]%
%
\>[2]\AgdaOperator{\AgdaInductiveConstructor{\AgdaUnderscore{}·\AgdaUnderscore{}}}%
\>[99I]\AgdaSymbol{:}\<%
\\
\>[0]\<%
\\
\>[.][@{}l@{}]\<[99I]%
\>[6]\AgdaSymbol{(}\AgdaBound{τ₁}\AgdaSpace{}%
\AgdaSymbol{:}\AgdaSpace{}%
\AgdaDatatype{Type}\AgdaSpace{}%
\AgdaBound{Δ}\AgdaSpace{}%
\AgdaSymbol{(}\AgdaGeneralizable{κ₁}\AgdaSpace{}%
\AgdaOperator{\AgdaInductiveConstructor{`→}}\AgdaSpace{}%
\AgdaGeneralizable{κ₂}\AgdaSymbol{))}\AgdaSpace{}%
\AgdaSymbol{→}\<%
\\
%
\>[6]\AgdaSymbol{(}\AgdaBound{τ₂}\AgdaSpace{}%
\AgdaSymbol{:}\AgdaSpace{}%
\AgdaDatatype{Type}\AgdaSpace{}%
\AgdaBound{Δ}\AgdaSpace{}%
\AgdaGeneralizable{κ₁}\AgdaSymbol{)}\AgdaSpace{}%
\AgdaSymbol{→}\<%
\\
%
\>[6]\AgdaDatatype{Type}\AgdaSpace{}%
\AgdaBound{Δ}\AgdaSpace{}%
\AgdaGeneralizable{κ₂}\<%
\\
%
\\[\AgdaEmptyExtraSkip]%
%
\>[2]\AgdaOperator{\AgdaInductiveConstructor{\AgdaUnderscore{}`→\AgdaUnderscore{}}}%
\>[114I]\AgdaSymbol{:}\<%
\\
%
\\[\AgdaEmptyExtraSkip]%
\>[114I][@{}l@{\AgdaIndent{0}}]%
\>[9]\AgdaSymbol{(}\AgdaBound{τ₁}\AgdaSpace{}%
\AgdaSymbol{:}\AgdaSpace{}%
\AgdaDatatype{Type}\AgdaSpace{}%
\AgdaBound{Δ}\AgdaSpace{}%
\AgdaInductiveConstructor{★}\AgdaSymbol{)}\AgdaSpace{}%
\AgdaSymbol{→}\<%
\\
%
\>[9]\AgdaSymbol{(}\AgdaBound{τ₂}\AgdaSpace{}%
\AgdaSymbol{:}\AgdaSpace{}%
\AgdaDatatype{Type}\AgdaSpace{}%
\AgdaBound{Δ}\AgdaSpace{}%
\AgdaInductiveConstructor{★}\AgdaSymbol{)}\AgdaSpace{}%
\AgdaSymbol{→}\<%
\\
%
\>[9]\AgdaDatatype{Type}\AgdaSpace{}%
\AgdaBound{Δ}\AgdaSpace{}%
\AgdaInductiveConstructor{★}\<%
\end{code}

Description, description, blah.

\begin{code}%
\>[0]\AgdaKeyword{data}\AgdaSpace{}%
\AgdaDatatype{Pred}\AgdaSpace{}%
\AgdaBound{Δ}\AgdaSpace{}%
\AgdaKeyword{where}\<%
\end{code}

\bibliographystyle{plainnat}
\bibliography{TN}
\end{document}
%%% Local Variables: 
%%% TeX-command-extra-options: "-shell-escape"
%%% End:
%  LocalWords:  denotational Agda Wadler dPoint sqrt subtyping coercions Intr
%  LocalWords:  RowTypes Bool eval GHC reified HillerstromL Leijen LindleyM RO
%  LocalWords:  ChapmanKNW Aydemir AbelAHPMSS AbelC AbelOV plfa HubersIMM STLC
%  LocalWords:  MorrisM denotationally DenotationalSoundness RowTheories Suc de
%  LocalWords:  ReifyingVariants RowTheory BerthomieuM CardelliMMS HarperP NatF
%  LocalWords:  XueOX GasterJ Sipser SaffrichTM Env Expr Agda's Leivant ChanW
%  LocalWords:  ThiemannW ImpredicativeSet ImpredicativeSetSucks AbelP chapman
%  LocalWords:  AltenkirchK KaposiKK Gaster XieOBS BiXOS Chlipala objTypes Bahr
%  LocalWords:  Garrigue KEnv PEnv
