\documentclass[authoryear, acmsmall, screen, review, nonacm]{acmart} % % use acmtog for two-column
\overfullrule=1mm
% \usepackage[margin=1.5in]{geometry}

\newcommand\Root{/home/alex/Dropbox/HubersPhD/bibliography}
\usepackage{
  enumitem,
  amsmath,
%  amsthm,
%  amssymb,
  xspace,
  mathtools,
  multicol,
  tikz-cd,
  parskip,
  amsfonts,
  mathrsfs,
  stmaryrd,
  tikz,
  float,
  titlecaps,
  soul,
  upgreek,
  subfiles,
  caption,
  graphicx,
  array,
  booktabs,
  subcaption,
  tikz,
  tikz-cd,
  centernot,
  ifthen,
  thmtools,
  fancyvrb,
  \Root/sty/mathwidth,
  \Root/sty/chronology,
  %\Root/sty/sectsty,
  \Root/sty/infer,
}

% -----------------------------------------------------------------------------
% Conditionals
\newif\ifstretch
\newif\ifdarkmode
\newif\ifcomments
\newif\iffastercode

\fastercodefalse
\commentstrue
\darkmodefalse
\stretchfalse

\ifdarkmode \pagecolor[rgb]{0,0,0} \color[rgb]{1,1,1} \fi

% -----------------------------------------------------------------------------
% My sty's
\usepackage{TN}

% -----------------------------------------------------------------------------
% ulem
\usepackage[normalem]{ulem}

% -----------------------------------------------------------------------------
% natbib
%\usepackage[square,numbers]{natbib}

% -----------------------------------------------------------------------------
% tikz
\usetikzlibrary{positioning}
\tikzset{main node/.style={circle,fill=white!20,draw,minimum size=1cm,inner sep=2pt}}

% -----------------------------------------------------------------------------
% Times New Roman, as per ACM.
%\usepackage{mathptmx}
%\DeclareMathAlphabet{\mathcal}{OMS}{cmsy}{m}{n}


% -----------------------------------------------------------------------------
% liststep and parindent
\setlist{nosep}
\setlist[enumerate]{nosep}
\setlength{\parindent}{10pt}


% -----------------------------------------------------------------------------
% Set counter to 0 (for abstract)
\setcounter{section}{0}


% -----------------------------------------------------------------------------
% Blank page
\newcommand\blankpage{% comando pagina vuota
    \clearpage
    \null
    \thispagestyle{empty}%
    \addtocounter{page}{-1}%
    \clearpage
}
% -----------------------------------------------------------------------------
% cleverref

\usepackage[capitalize,noabbrev]{cleveref}
\crefformat{section}{(\S#2#1#3)}
\crefformat{subsection}{(\S#2#1#3)}
\crefformat{subsubsection}{(\S#2#1#3)}
\crefmultiformat{section}{(\S\S#2#1#3}{ and~#2#1#3}{, #2#1#3}{, and~#2#1#3)}
\crefmultiformat{subsection}{(\S\S#2#1#3}{ and~#2#1#3}{, #2#1#3}{, and~#2#1#3)}
\crefmultiformat{subsubsection}{(\S\S#2#1#3}{ and~#2#1#3}{, #2#1#3}{, and~#2#1#3)}

% -----------------------------------------------------------------------------
% math envs


% \ifdefined\theorem \else
%   \newtheorem{theorem}{Theorem}
% \fi
% \newtheorem{assumption}[theorem]{Assumption}
% \newtheorem{prop}[theorem]{Proposition}
% \newtheorem{claim}[theorem]{Claim}
% \theoremstyle{definition}
% \newtheorem{defn}[theorem]{Definition}
% \ifdefined\example \else
%   \newtheorem{example}[theorem]{Example}
% \fi
% \theoremstyle{remark}
% \newtheorem*{remark}{Remark}

% \newcommand{\indextheorem}[1]{\index[theorems]{#1}} 
% \newcommand{\indexdefinition}[1]{\index[definitions]{#1}}

% \AtEndPreamble{%
%   \theoremstyle{acmplain}
%   %\newtheorem{theorem}{Theorem}
%   \newtheorem*{theorem*}{Theorem}}

% \newenvironment{AgdaDefs}%
%   {\begin{smalle}\[\begin{array}{l@{\;}c@{\;}l} }%
%   {\end{array}\]\end{smalle}\ignorespacesafterend}

% \newenvironment{AgdaDefsData}%
%   {\begin{smalle}\[\begin{array}{l@{\;}l@{\;}c@{\;}l} }%
%   {\end{array}\]\end{smalle}\ignorespacesafterend}
% %%% Local Variables: 
% %%% TeX-command-extra-options: "-shell-escape"
% %%% End: 
\usepackage{agda}
\usepackage[utf8]{inputenc}
\usepackage[T1]{fontenc}

% ACM garbage
\setcopyright{none}
\citestyle{acmauthoryear}
\settopmatter{printacmref=false, printfolios=true}
\renewcommand{\footnotetextcopyrightpermission}{} 
% Redefine the \acmDOI command to do nothing 
\pagestyle{empty}
\fancyfoot{}

%\usepackage[utf8]{inputenc}
% \numberwithin{equation}{section}
% \numberwithin{theorem}{section}

\title{Normalization By Evaluation of Types in \Rome}
\author{Alex Hubers}
\orcid{0000-0002-6237-3326}
\affiliation{
  \department{Department of Computer Science}
  \institution{The University of Iowa}
  \streetaddress{14 MacLean Hall}
  \city{Iowa City}
  \state{Iowa}
  \country{USA}}
\email{alexander-hubers@uiowa.edu}

\usepackage{newunicodechar}
\newunicodechar{∋}{$\ni$}
\newunicodechar{ε}{$\epsilon$}
\newunicodechar{·}{$\cdot$}
\newunicodechar{⊢}{$\vdash$}
\newunicodechar{⋆}{${}^\star$}
\newunicodechar{Π}{$\Pi$}
\newunicodechar{⇒}{$\Rightarrow$}
\newunicodechar{ƛ}{$\lambdabar$}
\newunicodechar{∅}{$\emptyset$}
\newunicodechar{∀}{$\forall$}
\newunicodechar{ϕ}{$\Phi$}
\newunicodechar{φ}{$\phi$}
\newunicodechar{ψ}{$\Psi$}
\newunicodechar{ρ}{$\rho$}
\newunicodechar{α}{$\alpha$}
\newunicodechar{β}{$\beta$}
\newunicodechar{μ}{$\mu$}
\newunicodechar{σ}{$\sigma$}
\newunicodechar{≡}{$\equiv$}
\newunicodechar{Γ}{$\Gamma$}
\newunicodechar{∥}{$\parallel$}
\newunicodechar{Λ}{$\Lambda$}
\newunicodechar{₂}{$_2$}
\newunicodechar{θ}{$\theta$}
\newunicodechar{Θ}{$\Theta$}
\newunicodechar{∘}{$\circ$}
\newunicodechar{Δ}{$\Delta$}
\newunicodechar{★}{$\star$}
\newunicodechar{λ}{$\lambda$}
\newunicodechar{⊧}{$\models$}
\newunicodechar{⊎}{$\uplus$}
\newunicodechar{η}{$\eta$}
\newunicodechar{⊥}{$\bot$}
\newunicodechar{Σ}{$\Sigma$}
\newunicodechar{ξ}{$\xi$}
\newunicodechar{₁}{$_1$}
\newunicodechar{ₖ}{$_k$}
\newunicodechar{₃}{$_3$}
\newunicodechar{ℕ}{$\mathbb{N}$}
\newunicodechar{ᶜ}{${}^c$}
\newunicodechar{Φ}{$\Phi$}
\newunicodechar{Ψ}{$\Psi$}
\newunicodechar{⊤}{$\top$}
\newunicodechar{κ}{$\kappa$}
\newunicodechar{τ}{$\tau$}
\newunicodechar{π}{$\pi$}
\newunicodechar{⌊}{$\lfloor$}
\newunicodechar{⌋}{$\rfloor$}
\newunicodechar{≲}{$\lesssim$}
\newunicodechar{▹}{$\triangleright$}
\newunicodechar{ℓ}{$\ell$}
\newunicodechar{υ}{$\upsilon$}

\newunicodechar{→}{$\rightarrow$}
\newunicodechar{×}{$\times$}
\newunicodechar{ω}{$\omega$}
\newunicodechar{∃}{$\exists$}
\newunicodechar{∈}{$\in$}
\newunicodechar{⇑}{$\Uparrow$}
\newunicodechar{⇓}{$\Downarrow$}
\newunicodechar{≋}{$\approx$}
\newunicodechar{ₗ}{$_l$}
\newunicodechar{ᵣ}{$_r$}
\newunicodechar{⟦}{$\llbracket$}
\newunicodechar{⟧}{$\rrbracket$}
\newunicodechar{⁻}{$^{-}$}
\newunicodechar{¹}{$^{1}$}
\newunicodechar{₄}{$_{4}$}
\newunicodechar{⦅}{$\llparenthesis$}
\newunicodechar{⦆}{$\rrparenthesis$}
\newunicodechar{─}{$\setminus$}
\newunicodechar{∷}{$\co\co$}
\newunicodechar{ₖ}{$_{k}$}
\newunicodechar{ₙ}{$_{n}$}
\newunicodechar{≟}{$\overset{?}{=}$}
\newunicodechar{𝒯}{$\mathcal T$}
\newunicodechar{⨾}{$\co$}
\newunicodechar{Ξ}{$\Xi$}
\newunicodechar{ξ}{$\xi$}

\begin{document}

\maketitle

\section*{Abstract}
We describe the normalization-by-evaluation (NbE) of types in \Rome, a row calculus with recursive types, qualified types, and a novel \emph{row complement} operator. Types are normalized to $\beta\eta$-long forms modulo a type equivalence relation. Because the type system of \Rome is a strict extension of System \Fome, much of the type reduction is isomorphic to reduction of terms in the STLC. Novel to this report are the reductions of row, record, and variant types.

\section{The \Rome{} calculus}

For reference, \cref{fig:syntax-types} describes the syntax of kinds, predicates, and types in \Rome. We forego further description to the next section.

\begin{figure}[H]
\begin{gather*}
\begin{array}{r@{\hspace{5px}}l@{\qquad}r@{\hspace{5px}}l@{\qquad}r@{\hspace{5px}}l@{\qquad}r@{\hspace{5px}}l}
\text{Type variables} & \alpha \in \mathcal A & \text{Labels} & \ell \in \mathcal L
\end{array}
\\[5px]
\begin{doublesyntaxarray}
  \mcl{\text{Kinds}} & \kappa & ::= & \TypeK \mid \LabK \mid \RowK \kappa \mid \kappa \to \kappa \\
  \mcl{\text{Predicates}} & \pi, \psi & ::= & \LeqP \rho \rho \mid \PlusP \rho \rho \rho \\
  \text{Types} & \mcr{\Types \ni \phi, \tau, \upsilon, \rho, \xi} & ::= & \alpha \mid \pi \then \tau \mid \forall \alpha\co\kappa. \tau \mid \lambda \alpha\co\kappa. \tau \mid \tau \, \tau \\
               &                              &     & \mid    & \RowIx i 0 m {\LabTy {\xi_i} {\tau_i}} \mid \ell \mid \Sing{\tau} \mid \Mapp{\phi}{\rho} \mid \rho \Compl \rho \\ 
               &                              &     & \mid & \tau \to \tau \mid \Pi \mid \Sigma \mid \mu \, \phi 
\end{doublesyntaxarray}
\end{gather*}
\caption{Syntax}
\label{fig:syntax-types}
\end{figure}

\subsection{Example types and the need for reduction}
% Add motivation for where this incurs type normalization. Show "hidden" maps and hidden reductions. How does $(\Pi\, (\ell \triangleright f)) \, \tau$ reduce?
We will write Rome types in programs in the slightly-altered syntax of \emph{Rosi}, our experimental implementation of \Rome. 
Wand's problem:

\begin{rosi}
wand : forall l x y z t. x + y ~ z, {l := t} < z => #l -> Pi x -> Pi y -> t
\end{rosi}

Here we can simulate the deriving of functor typeclass instances: given a record of \verb!fmap! instances, I can give you a \verb!Functor! instance for $\Sigma\, z$.

\begin{rosi}
type Functor : (* -> *) -> *
type Functor = \f. forall a b. (a -> b) -> f a -> f b

fmapS : forall z : R[* -> *]. Pi (Functor z) -> Functor (Sigma z)
\end{rosi}

Let's first elaborate this type by rendering some implicit notation (e.g., maps) explicit.

\begin{rosi}
fmapS : forall z : R[* -> *]. Pi (Functor z) -> Functor (Sigma z)
\end{rosi}

\Ni And a desugaring of booleans to Church encodings:

\begin{rosi}
desugar : forall y. BoolF < y, LamF < y - BoolF =>
          Pi (Functor (y - BoolF)) -> Mu (Sigma y) -> Mu (Sigma (y - BoolF))
\end{rosi}

\section{Type Reduction}
\subsection{Normal forms}

By directing the type equivalence relation we define computation on types. This serves as a sort of specification on the shape normal forms of types ought to have. Our grammar for normal types must be carefully crafted so as to be neither too "large" nor too "small". In particular, we wish our normalization algorithm to be \emph{stable}, which implies surjectivity. Hence if the normal syntax is too large---i.e., it produces junk types---then these junk types will have pre-images in the domain of normalization. Inversely, if the normal syntax is too small, then there will be types whose normal forms cannot be expressed. \Cref{fig:type-normalization} specifies the syntax and typing of normal types, given as reference. We describe the syntax in more depth by describing its intrinsic mechanization.

\begin{figure}[H]
\begin{gather*}
\begin{array}{r@{\hspace{7px}}l@{\qquad\qquad}r@{\hspace{7px}}l}
  \text{Type variables} & \alpha \in \mathcal A &
  \text{Labels} & \ell \in \mathcal L
\end{array} \\
\begin{doublesyntaxarray}
  \mcl{\text{Ground Kinds}}  & \gamma   & ::= & \TypeK \mid \LabK \\
  \mcl{\text{Kinds}}         & \kappa    & ::= & \gamma \mid \kappa \to \kappa \mid  \RowK \kappa \\
  \mcl{\text{Row Literals}}   & \NormalRows \ni \Normal \rho    & ::= & \RowIx i 0 m {\LabTy {\ell_i} {\Normal {\tau_i}}} \\
  \mcl{\text{Neutral Types}} & n    & ::= & \alpha \mid n \, {\Normal \tau}  \\
  \mcl{\text{Normal Types}}  & \NormalTypes \ni \Normal \tau, \Normal \phi & ::= & n \mid \Mapp {\hat{\phi}} {n} \mid \Normal{\rho} \mid \Normal{\pi} \then \Normal{\tau} \mid \forall \alpha\co\kappa. \Normal{\tau} \mid \lambda \alpha\co\kappa. \Normal{\tau} \\
                             &       &     & \mid & \LabTy n {\Normal \tau} \mid \ell \mid \Sing {\Normal \tau} \mid {\Normal \tau} \Compl {\Normal \tau} \mid \Pi \, {\Normal \tau} \mid \Sigma \, {\Normal \tau}  \\
\end{doublesyntaxarray}
\end{gather*}
\caption{Normal type forms}
\label{fig:type-normalization}
\end{figure}

\subsection{Metatheory}
\subsubsection{Canonicity of normal types}

The syntax of normal types is defined precisely so as to enjoy canonical forms
based on kind.
\subsubsection{Completeness of normalization}
\subsubsection{Soundness of normalization}
\subsubsection{Decidability of type conversion}

\section{Normalization by Evaluation (NbE)}
\begin{code}[hide]%
\>[0]\AgdaKeyword{postulate}\<%
\\
\>[0][@{}l@{\AgdaIndent{0}}]%
\>[2]\AgdaPostulate{bot}\AgdaSpace{}%
\AgdaSymbol{:}\AgdaSpace{}%
\AgdaSymbol{∀}\AgdaSpace{}%
\AgdaSymbol{(}\AgdaBound{X}\AgdaSpace{}%
\AgdaSymbol{:}\AgdaSpace{}%
\AgdaPrimitive{Set}\AgdaSymbol{)}\AgdaSpace{}%
\AgdaSymbol{→}\AgdaSpace{}%
\AgdaBound{X}\<%
\\
\>[0]\AgdaKeyword{open}\AgdaSpace{}%
\AgdaKeyword{import}\AgdaSpace{}%
\AgdaModule{Rome.Kinds.Syntax}\<%
\\
\>[0]\AgdaKeyword{open}\AgdaSpace{}%
\AgdaKeyword{import}\AgdaSpace{}%
\AgdaModule{Rome.Kinds.GVars}\<%
\\
\>[0]\AgdaKeyword{open}\AgdaSpace{}%
\AgdaKeyword{import}\AgdaSpace{}%
\AgdaModule{Rome.Types.Syntax}\<%
\\
\>[0]\AgdaKeyword{open}\AgdaSpace{}%
\AgdaKeyword{import}\AgdaSpace{}%
\AgdaModule{Rome.Types.Normal.Syntax}\<%
\\
\>[0]\AgdaKeyword{open}\AgdaSpace{}%
\AgdaKeyword{import}\AgdaSpace{}%
\AgdaModule{Rome.Types.Semantic.Syntax}\<%
\\
\>[0]\AgdaKeyword{open}\AgdaSpace{}%
\AgdaKeyword{import}\AgdaSpace{}%
\AgdaModule{Rome.Prelude}\<%
\end{code}

\subsection{The semantic domain}

\subsection{reflection \& reification}

\subsection{Evaluation}

\subsection{Normalization}
\begin{code}%
\>[0]\AgdaComment{--\ ⇓\ :\ ∀\ \{Δ\}\ →\ Type\ Δ\ κ\ →\ NormalType\ Δ\ κ}\<%
\\
\>[0]\AgdaComment{--\ ⇓\ τ\ =\ reify\ (eval\ τ\ idEnv)}\<%
\end{code}

\section{Mechanizing Metatheory}

\subsection{Stability}

\begin{code}%
\>[0]\AgdaComment{--\ stability\ \ \ :\ ∀\ (τ\ :\ NormalType\ Δ\ κ)\ →\ ⇓\ (⇑\ τ)\ ≡\ τ}\<%
\\
\>[0]\AgdaComment{--\ stabilityNE\ :\ ∀\ (τ\ :\ NeutralType\ Δ\ κ)\ →\ eval\ (⇑NE\ τ)\ (idEnv\ \{Δ\})\ ≡\ reflect\ τ}\<%
\\
\>[0]\AgdaComment{--\ stabilityPred\ :\ ∀\ (π\ :\ NormalPred\ Δ\ R[\ κ\ ])\ →\ evalPred\ (⇑Pred\ π)\ idEnv\ ≡\ π}\<%
\\
\>[0]\AgdaComment{--\ stabilityRow\ :\ ∀\ (ρ\ :\ SimpleRow\ NormalType\ Δ\ R[\ κ\ ])\ →\ reifyRow\ (evalRow\ (⇑Row\ ρ)\ idEnv)\ ≡\ ρ}\<%
\end{code}
\begin{code}[hide]%
\>[0]\AgdaComment{--\ stability\ \ \ \ \ =\ bot\ \AgdaUnderscore{}}\<%
\\
\>[0]\AgdaComment{--\ stabilityNE\ \ \ =\ bot\ \AgdaUnderscore{}}\<%
\\
\>[0]\AgdaComment{--\ stabilityPred\ =\ bot\ \AgdaUnderscore{}}\<%
\\
\>[0]\AgdaComment{--\ stabilityRow\ =\ bot\ \AgdaUnderscore{}}\<%
\end{code}

Stability implies surjectivity and idempotency.

\begin{code}%
\>[0]\AgdaComment{--\ idempotency\ :\ ∀\ (τ\ :\ Type\ Δ\ κ)\ →\ (⇑\ ∘\ ⇓\ ∘\ ⇑\ ∘\ ⇓)\ τ\ ≡\ \ (⇑\ ∘\ ⇓)\ \ τ}\<%
\\
\>[0]\AgdaComment{--\ idempotency\ τ\ rewrite\ stability\ (⇓\ τ)\ =\ refl}\<%
\\
%
\\[\AgdaEmptyExtraSkip]%
\>[0]\AgdaComment{--\ surjectivity\ :\ ∀\ (τ\ :\ NormalType\ Δ\ κ)\ →\ ∃[\ υ\ ]\ (⇓\ υ\ ≡\ τ)}\<%
\\
\>[0]\AgdaComment{--\ surjectivity\ τ\ =\ (\ ⇑\ τ\ ,\ stability\ τ\ )\ }\<%
\end{code}

Dual to surjectivity, stability also implies that embedding is injective.
 
\begin{code}%
\>[0]\AgdaComment{--\ ⇑-inj\ :\ ∀\ (τ₁\ τ₂\ :\ NormalType\ Δ\ κ)\ →\ ⇑\ τ₁\ ≡\ ⇑\ τ₂\ →\ τ₁\ ≡\ τ₂\ \ \ \ \ \ \ \ \ \ \ \ \ \ \ \ \ \ \ }\<%
\\
\>[0]\AgdaComment{--\ ⇑-inj\ τ₁\ τ₂\ eq\ =\ trans\ (sym\ (stability\ τ₁))\ (trans\ (cong\ ⇓\ eq)\ (stability\ τ₂))}\<%
\end{code}

\subsection{A logical relation for completeness}

\begin{code}%
\>[0]\AgdaComment{--\ subst-Row\ :\ ∀\ \{A\ :\ Set\}\ \{n\ m\ :\ ℕ\}\ →\ (n\ ≡\ m)\ →\ (f\ :\ Fin\ n\ →\ A)\ →\ Fin\ m\ →\ A\ }\<%
\\
\>[0]\AgdaComment{--\ subst-Row\ refl\ f\ =\ f}\<%
\\
%
\\[\AgdaEmptyExtraSkip]%
\>[0]\AgdaComment{--\ --\ Completeness\ relation\ on\ semantic\ types}\<%
\\
\>[0]\AgdaComment{--\ \AgdaUnderscore{}≋\AgdaUnderscore{}\ :\ SemType\ Δ\ κ\ →\ SemType\ Δ\ κ\ →\ Set}\<%
\\
\>[0]\AgdaComment{--\ \AgdaUnderscore{}≋₂\AgdaUnderscore{}\ :\ ∀\ \{A\}\ →\ (x\ y\ :\ A\ ×\ SemType\ Δ\ κ)\ →\ Set}\<%
\\
\>[0]\AgdaComment{--\ (l₁\ ,\ τ₁)\ ≋₂\ (l₂\ ,\ τ₂)\ =\ l₁\ ≡\ l₂\ ×\ τ₁\ ≋\ τ₂}\<%
\\
\>[0]\AgdaComment{--\ \AgdaUnderscore{}≋R\AgdaUnderscore{}\ :\ (ρ₁\ ρ₂\ :\ Row\ (SemType\ Δ\ κ))\ →\ Set\ }\<%
\\
\>[0]\AgdaComment{--\ (n\ ,\ P)\ ≋R\ (m\ ,\ Q)\ =\ Σ[\ pf\ ∈\ (n\ ≡\ m)\ ]\ (∀\ (i\ :\ Fin\ m)\ →\ \ (subst-Row\ pf\ P)\ i\ ≋₂\ Q\ i)}\<%
\\
%
\\[\AgdaEmptyExtraSkip]%
\>[0]\AgdaComment{--\ PointEqual-≋\ :\ ∀\ \{Δ₁\}\ \{κ₁\}\ \{κ₂\}\ (F\ G\ :\ KripkeFunction\ Δ₁\ κ₁\ κ₂)\ →\ Set}\<%
\\
\>[0]\AgdaComment{--\ PointEqualNE-≋\ :\ ∀\ \{Δ₁\}\ \{κ₁\}\ \{κ₂\}\ (F\ G\ :\ KripkeFunctionNE\ Δ₁\ κ₁\ κ₂)\ →\ Set}\<%
\\
\>[0]\AgdaComment{--\ Uniform\ :\ \ ∀\ \{Δ\}\ \{κ₁\}\ \{κ₂\}\ →\ KripkeFunction\ Δ\ κ₁\ κ₂\ →\ Set}\<%
\\
\>[0]\AgdaComment{--\ UniformNE\ :\ \ ∀\ \{Δ\}\ \{κ₁\}\ \{κ₂\}\ →\ KripkeFunctionNE\ Δ\ κ₁\ κ₂\ →\ Set}\<%
\\
%
\\[\AgdaEmptyExtraSkip]%
\>[0]\AgdaComment{--\ \AgdaUnderscore{}≋\AgdaUnderscore{}\ \{κ\ =\ ★\}\ τ₁\ τ₂\ =\ τ₁\ ≡\ τ₂}\<%
\\
\>[0]\AgdaComment{--\ \AgdaUnderscore{}≋\AgdaUnderscore{}\ \{κ\ =\ L\}\ τ₁\ τ₂\ =\ τ₁\ ≡\ τ₂}\<%
\\
\>[0]\AgdaComment{--\ \AgdaUnderscore{}≋\AgdaUnderscore{}\ \{Δ₁\}\ \{κ\ =\ κ₁\ `→\ κ₂\}\ F\ G\ =\ }\<%
\\
\>[0]\AgdaComment{--\ \ \ Uniform\ F\ ×\ Uniform\ G\ ×\ PointEqual-≋\ \{Δ₁\}\ F\ G\ }\<%
\\
\>[0]\AgdaComment{--\ \AgdaUnderscore{}≋\AgdaUnderscore{}\ \{Δ₁\}\ \{R[\ κ₂\ ]\}\ (\AgdaUnderscore{}<\$>\AgdaUnderscore{}\ \{κ₁\}\ φ₁\ n₁)\ (\AgdaUnderscore{}<\$>\AgdaUnderscore{}\ \{κ₁'\}\ φ₂\ n₂)\ =\ }\<%
\\
\>[0]\AgdaComment{--\ \ \ Σ[\ pf\ ∈\ (κ₁\ ≡\ κ₁')\ ]\ \ }\<%
\\
\>[0]\AgdaComment{--\ \ \ \ \ UniformNE\ φ₁}\<%
\\
\>[0]\AgdaComment{--\ \ \ ×\ UniformNE\ φ₂}\<%
\\
\>[0]\AgdaComment{--\ \ \ ×\ (PointEqualNE-≋\ (convKripkeNE₁\ pf\ φ₁)\ φ₂}\<%
\\
\>[0]\AgdaComment{--\ \ \ ×\ convNE\ pf\ n₁\ ≡\ n₂)}\<%
\\
\>[0]\AgdaComment{--\ \AgdaUnderscore{}≋\AgdaUnderscore{}\ \{Δ₁\}\ \{R[\ κ₂\ ]\}\ (φ₁\ <\$>\ n₁)\ \AgdaUnderscore{}\ =\ ⊥}\<%
\\
\>[0]\AgdaComment{--\ \AgdaUnderscore{}≋\AgdaUnderscore{}\ \{Δ₁\}\ \{R[\ κ₂\ ]\}\ \AgdaUnderscore{}\ (φ₁\ <\$>\ n₁)\ =\ ⊥}\<%
\\
\>[0]\AgdaComment{--\ \AgdaUnderscore{}≋\AgdaUnderscore{}\ \{Δ₁\}\ \{R[\ κ\ ]\}\ (l₁\ ▹\ τ₁)\ (l₂\ ▹\ τ₂)\ =\ l₁\ ≡\ l₂\ ×\ τ₁\ ≋\ τ₂}\<%
\\
\>[0]\AgdaComment{--\ \AgdaUnderscore{}≋\AgdaUnderscore{}\ \{Δ₁\}\ \{R[\ κ\ ]\}\ (x₁\ ▹\ x₂)\ (row\ ρ\ x₃)\ =\ ⊥}\<%
\\
\>[0]\AgdaComment{--\ \AgdaUnderscore{}≋\AgdaUnderscore{}\ \{Δ₁\}\ \{R[\ κ\ ]\}\ (x₁\ ▹\ x₂)\ (ρ₂\ ─\ ρ₃)\ =\ ⊥}\<%
\\
\>[0]\AgdaComment{--\ \AgdaUnderscore{}≋\AgdaUnderscore{}\ \{Δ₁\}\ \{R[\ κ\ ]\}\ (row\ ρ\ x₁)\ (x₂\ ▹\ x₃)\ =\ ⊥}\<%
\\
\>[0]\AgdaComment{--\ \AgdaUnderscore{}≋\AgdaUnderscore{}\ \{Δ₁\}\ \{R[\ κ\ ]\}\ (row\ (n\ ,\ P)\ x₁)\ (row\ (m\ ,\ Q)\ x₂)\ =\ (n\ ,\ P)\ ≋R\ (m\ ,\ Q)}\<%
\\
\>[0]\AgdaComment{--\ \AgdaUnderscore{}≋\AgdaUnderscore{}\ \{Δ₁\}\ \{R[\ κ\ ]\}\ (row\ ρ\ x₁)\ (ρ₂\ ─\ ρ₃)\ =\ ⊥}\<%
\\
\>[0]\AgdaComment{--\ \AgdaUnderscore{}≋\AgdaUnderscore{}\ \{Δ₁\}\ \{R[\ κ\ ]\}\ (ρ₁\ ─\ ρ₂)\ (x₁\ ▹\ x₂)\ =\ ⊥}\<%
\\
\>[0]\AgdaComment{--\ \AgdaUnderscore{}≋\AgdaUnderscore{}\ \{Δ₁\}\ \{R[\ κ\ ]\}\ (ρ₁\ ─\ ρ₂)\ (row\ ρ\ x₁)\ =\ ⊥}\<%
\\
\>[0]\AgdaComment{--\ \AgdaUnderscore{}≋\AgdaUnderscore{}\ \{Δ₁\}\ \{R[\ κ\ ]\}\ (ρ₁\ ─\ ρ₂)\ (ρ₃\ ─\ ρ₄)\ =\ ρ₁\ ≋\ ρ₃\ ×\ ρ₂\ ≋\ ρ₄}\<%
\\
%
\\[\AgdaEmptyExtraSkip]%
\>[0]\AgdaComment{--\ PointEqual-≋\ \{Δ₁\}\ \{κ₁\}\ \{κ₂\}\ F\ G\ =\ }\<%
\\
\>[0]\AgdaComment{--\ \ \ ∀\ \{Δ₂\}\ (ρ\ :\ Renamingₖ\ Δ₁\ Δ₂)\ \{V₁\ V₂\ :\ SemType\ Δ₂\ κ₁\}\ →\ }\<%
\\
\>[0]\AgdaComment{--\ \ \ V₁\ ≋\ V₂\ →\ F\ ρ\ V₁\ ≋\ G\ ρ\ V₂}\<%
\\
%
\\[\AgdaEmptyExtraSkip]%
\>[0]\AgdaComment{--\ PointEqualNE-≋\ \{Δ₁\}\ \{κ₁\}\ \{κ₂\}\ F\ G\ =\ }\<%
\\
\>[0]\AgdaComment{--\ \ \ ∀\ \{Δ₂\}\ (ρ\ :\ Renamingₖ\ Δ₁\ Δ₂)\ (V\ :\ NeutralType\ Δ₂\ κ₁)\ →\ }\<%
\\
\>[0]\AgdaComment{--\ \ \ F\ ρ\ V\ ≋\ G\ ρ\ V}\<%
\\
%
\\[\AgdaEmptyExtraSkip]%
\>[0]\AgdaComment{--\ Uniform\ \{Δ₁\}\ \{κ₁\}\ \{κ₂\}\ F\ =\ }\<%
\\
\>[0]\AgdaComment{--\ \ \ ∀\ \{Δ₂\ Δ₃\}\ (ρ₁\ :\ Renamingₖ\ Δ₁\ Δ₂)\ (ρ₂\ :\ Renamingₖ\ Δ₂\ Δ₃)\ (V₁\ V₂\ :\ SemType\ Δ₂\ κ₁)\ →}\<%
\\
\>[0]\AgdaComment{--\ \ \ V₁\ ≋\ V₂\ →\ (renSem\ ρ₂\ (F\ ρ₁\ V₁))\ ≋\ (renKripke\ ρ₁\ F\ ρ₂\ (renSem\ ρ₂\ V₂))}\<%
\\
%
\\[\AgdaEmptyExtraSkip]%
\>[0]\AgdaComment{--\ UniformNE\ \{Δ₁\}\ \{κ₁\}\ \{κ₂\}\ F\ =\ }\<%
\\
\>[0]\AgdaComment{--\ \ \ ∀\ \{Δ₂\ Δ₃\}\ (ρ₁\ :\ Renamingₖ\ Δ₁\ Δ₂)\ (ρ₂\ :\ Renamingₖ\ Δ₂\ Δ₃)\ (V\ :\ NeutralType\ Δ₂\ κ₁)\ →}\<%
\\
\>[0]\AgdaComment{--\ \ \ (renSem\ ρ₂\ (F\ ρ₁\ V))\ ≋\ F\ (ρ₂\ ∘\ ρ₁)\ (renₖNE\ ρ₂\ V)}\<%
\\
%
\\[\AgdaEmptyExtraSkip]%
\>[0]\AgdaComment{--\ Env-≋\ :\ (η₁\ η₂\ :\ Env\ Δ₁\ Δ₂)\ →\ Set}\<%
\\
\>[0]\AgdaComment{--\ Env-≋\ η₁\ η₂\ =\ ∀\ \{κ\}\ (x\ :\ TVar\ \AgdaUnderscore{}\ κ)\ →\ (η₁\ x)\ ≋\ (η₂\ x)}\<%
\end{code}

\subsubsection{Properties}~

\begin{code}%
\>[0]\AgdaComment{--\ reflect-≋\ \ :\ ∀\ \{τ₁\ τ₂\ :\ NeutralType\ Δ\ κ\}\ →\ τ₁\ ≡\ τ₂\ →\ reflect\ τ₁\ ≋\ reflect\ τ₂}\<%
\\
\>[0]\AgdaComment{--\ reify-≋\ \ \ \ :\ ∀\ \{V₁\ V₂\ :\ SemType\ Δ\ κ\}\ \ \ \ \ →\ V₁\ ≋\ V₂\ →\ reify\ V₁\ \ \ ≡\ reify\ V₂\ }\<%
\\
\>[0]\AgdaComment{--\ reifyRow-≋\ :\ ∀\ \{n\}\ (P\ Q\ :\ Fin\ n\ →\ Label\ ×\ SemType\ Δ\ κ)\ →\ \ }\<%
\\
\>[0]\AgdaComment{--\ \ \ \ \ \ \ \ \ \ \ \ \ \ \ \ (∀\ (i\ :\ Fin\ n)\ →\ P\ i\ ≋₂\ Q\ i)\ →\ }\<%
\\
\>[0]\AgdaComment{--\ \ \ \ \ \ \ \ \ \ \ \ \ \ \ \ reifyRow\ (n\ ,\ P)\ ≡\ reifyRow\ (n\ ,\ Q)}\<%
\end{code}
\begin{code}[hide]%
\>[0]\AgdaComment{--\ reflect-≋\ \ =\ bot\ \AgdaUnderscore{}\ }\<%
\\
\>[0]\AgdaComment{--\ reify-≋\ \ \ \ =\ bot\ \AgdaUnderscore{}\ }\<%
\\
\>[0]\AgdaComment{--\ reifyRow-≋\ =\ bot\ \AgdaUnderscore{}\ }\<%
\end{code}

\subsection{The fundamental theorem and completeness}

\begin{code}%
\>[0]\AgdaComment{--\ fundC\ :\ ∀\ \{τ₁\ τ₂\ :\ Type\ Δ₁\ κ\}\ \{η₁\ η₂\ :\ Env\ Δ₁\ Δ₂\}\ →\ }\<%
\\
\>[0]\AgdaComment{--\ \ \ \ \ \ \ \ Env-≋\ η₁\ η₂\ →\ τ₁\ ≡t\ τ₂\ →\ eval\ τ₁\ η₁\ ≋\ eval\ τ₂\ η₂}\<%
\\
\>[0]\AgdaComment{--\ fundC-pred\ :\ ∀\ \{π₁\ π₂\ :\ Pred\ Type\ Δ₁\ R[\ κ\ ]\}\ \{η₁\ η₂\ :\ Env\ Δ₁\ Δ₂\}\ →\ }\<%
\\
\>[0]\AgdaComment{--\ \ \ \ \ \ \ \ \ \ \ \ \ Env-≋\ η₁\ η₂\ →\ π₁\ ≡p\ π₂\ →\ evalPred\ π₁\ η₁\ ≡\ evalPred\ π₂\ η₂}\<%
\\
\>[0]\AgdaComment{--\ fundC-Row\ :\ ∀\ \{ρ₁\ ρ₂\ :\ SimpleRow\ Type\ Δ₁\ R[\ κ\ ]\}\ \{η₁\ η₂\ :\ Env\ Δ₁\ Δ₂\}\ →\ }\<%
\\
\>[0]\AgdaComment{--\ \ \ \ \ \ \ \ \ \ \ \ \ Env-≋\ η₁\ η₂\ →\ ρ₁\ ≡r\ ρ₂\ →\ evalRow\ ρ₁\ η₁\ ≋R\ evalRow\ ρ₂\ η₂}\<%
\end{code}
\begin{code}[hide]%
\>[0]\AgdaComment{--\ fundC\ =\ bot\ \AgdaUnderscore{}}\<%
\\
\>[0]\AgdaComment{--\ fundC-pred\ =\ bot\ \AgdaUnderscore{}}\<%
\\
\>[0]\AgdaComment{--\ fundC-Row\ =\ bot\ \AgdaUnderscore{}}\<%
\end{code}

\begin{code}%
\>[0]\AgdaComment{--\ idEnv-≋\ :\ ∀\ \{Δ\}\ →\ Env-≋\ (idEnv\ \{Δ\})\ (idEnv\ \{Δ\})}\<%
\\
\>[0]\AgdaComment{--\ idEnv-≋\ x\ =\ reflect-≋\ refl}\<%
\\
%
\\[\AgdaEmptyExtraSkip]%
\>[0]\AgdaComment{--\ completeness\ :\ ∀\ \{τ₁\ τ₂\ :\ Type\ Δ\ κ\}\ →\ τ₁\ ≡t\ τ₂\ →\ ⇓\ τ₁\ ≡\ ⇓\ τ₂}\<%
\\
\>[0]\AgdaComment{--\ completeness\ eq\ =\ reify-≋\ (fundC\ idEnv-≋\ eq)\ \ }\<%
\\
%
\\[\AgdaEmptyExtraSkip]%
\>[0]\AgdaComment{--\ completeness-row\ :\ ∀\ \{ρ₁\ ρ₂\ :\ SimpleRow\ Type\ Δ\ R[\ κ\ ]\}\ →\ ρ₁\ ≡r\ ρ₂\ →\ ⇓Row\ ρ₁\ ≡\ ⇓Row\ ρ₂}\<%
\end{code}
\begin{code}[hide]%
\>[0]\AgdaComment{--\ completeness-row\ =\ bot\ \AgdaUnderscore{}}\<%
\end{code}

\subsection{A logical relation for soundness}
\begin{code}%
\>[0]\AgdaComment{--\ infix\ 0\ ⟦\AgdaUnderscore{}⟧≋\AgdaUnderscore{}}\<%
\\
\>[0]\AgdaComment{--\ ⟦\AgdaUnderscore{}⟧≋\AgdaUnderscore{}\ :\ ∀\ \{κ\}\ →\ Type\ Δ\ κ\ →\ SemType\ Δ\ κ\ →\ Set}\<%
\\
\>[0]\AgdaComment{--\ ⟦\AgdaUnderscore{}⟧≋ne\AgdaUnderscore{}\ :\ ∀\ \{κ\}\ →\ Type\ Δ\ κ\ →\ NeutralType\ Δ\ κ\ →\ Set}\<%
\\
%
\\[\AgdaEmptyExtraSkip]%
\>[0]\AgdaComment{--\ ⟦\AgdaUnderscore{}⟧r≋\AgdaUnderscore{}\ :\ ∀\ \{κ\}\ →\ SimpleRow\ Type\ Δ\ R[\ κ\ ]\ →\ Row\ (SemType\ Δ\ κ)\ →\ Set}\<%
\\
\>[0]\AgdaComment{--\ ⟦\AgdaUnderscore{}⟧≋₂\AgdaUnderscore{}\ :\ ∀\ \{κ\}\ →\ Label\ ×\ Type\ Δ\ κ\ →\ Label\ ×\ SemType\ Δ\ κ\ →\ Set}\<%
\\
\>[0]\AgdaComment{--\ ⟦\ (l₁\ ,\ τ)\ ⟧≋₂\ (l₂\ ,\ V)\ =\ (l₁\ ≡\ l₂)\ ×\ (⟦\ τ\ ⟧≋\ V)}\<%
\\
%
\\[\AgdaEmptyExtraSkip]%
\>[0]\AgdaComment{--\ SoundKripke\ :\ Type\ Δ₁\ (κ₁\ `→\ κ₂)\ →\ KripkeFunction\ Δ₁\ κ₁\ κ₂\ →\ Set}\<%
\\
\>[0]\AgdaComment{--\ SoundKripkeNE\ :\ Type\ Δ₁\ (κ₁\ `→\ κ₂)\ →\ KripkeFunctionNE\ Δ₁\ κ₁\ κ₂\ →\ Set}\<%
\\
%
\\[\AgdaEmptyExtraSkip]%
\>[0]\AgdaComment{--\ --\ τ\ is\ equivalent\ to\ neutral\ `n`\ if\ it's\ equivalent\ }\<%
\\
\>[0]\AgdaComment{--\ --\ to\ the\ η\ and\ map-id\ expansion\ of\ n}\<%
\\
\>[0]\AgdaComment{--\ ⟦\AgdaUnderscore{}⟧≋ne\AgdaUnderscore{}\ τ\ n\ =\ τ\ ≡t\ ⇑\ (η-norm\ n)}\<%
\\
%
\\[\AgdaEmptyExtraSkip]%
\>[0]\AgdaComment{--\ ⟦\AgdaUnderscore{}⟧≋\AgdaUnderscore{}\ \{κ\ =\ ★\}\ τ₁\ τ₂\ =\ τ₁\ ≡t\ ⇑\ τ₂}\<%
\\
\>[0]\AgdaComment{--\ ⟦\AgdaUnderscore{}⟧≋\AgdaUnderscore{}\ \{κ\ =\ L\}\ τ₁\ τ₂\ =\ τ₁\ ≡t\ ⇑\ τ₂}\<%
\\
\>[0]\AgdaComment{--\ ⟦\AgdaUnderscore{}⟧≋\AgdaUnderscore{}\ \{Δ₁\}\ \{κ\ =\ κ₁\ `→\ κ₂\}\ f\ F\ =\ SoundKripke\ f\ F}\<%
\\
\>[0]\AgdaComment{--\ ⟦\AgdaUnderscore{}⟧≋\AgdaUnderscore{}\ \{Δ\}\ \{κ\ =\ R[\ κ\ ]\}\ τ\ (row\ (n\ ,\ P)\ \ oρ)\ =}\<%
\\
\>[0]\AgdaComment{--\ \ \ \ \ let\ xs\ =\ ⇑Row\ (reifyRow\ (n\ ,\ P))\ in\ }\<%
\\
\>[0]\AgdaComment{--\ \ \ \ \ (τ\ ≡t\ ⦅\ xs\ ⦆\ (fromWitness\ (Ordered⇑\ (reifyRow\ (n\ ,\ P))\ (reifyRowOrdered'\ n\ P\ oρ))))\ ×\ }\<%
\\
\>[0]\AgdaComment{--\ \ \ \ \ (⟦\ xs\ ⟧r≋\ (n\ ,\ P))}\<%
\\
\>[0]\AgdaComment{--\ ⟦\AgdaUnderscore{}⟧≋\AgdaUnderscore{}\ \{Δ\}\ \{κ\ =\ R[\ κ\ ]\}\ τ\ (l\ ▹\ V)\ =\ (τ\ ≡t\ (⇑NE\ l\ ▹\ ⇑\ (reify\ V)))\ ×\ (⟦\ ⇑\ (reify\ V)\ ⟧≋\ V)}\<%
\\
\>[0]\AgdaComment{--\ ⟦\AgdaUnderscore{}⟧≋\AgdaUnderscore{}\ \{Δ\}\ \{κ\ =\ R[\ κ\ ]\}\ τ\ ((ρ₂\ ─\ ρ₁)\ \{nr\})\ =\ (τ\ ≡t\ (⇑\ (reify\ ((ρ₂\ ─\ ρ₁)\ \{nr\}))))\ ×\ (⟦\ ⇑\ (reify\ ρ₂)\ ⟧≋\ ρ₂)\ ×\ (⟦\ ⇑\ (reify\ ρ₁)\ ⟧≋\ ρ₁)}\<%
\\
\>[0]\AgdaComment{--\ ⟦\AgdaUnderscore{}⟧≋\AgdaUnderscore{}\ \{Δ\}\ \{κ\ =\ R[\ κ\ ]\}\ τ\ (φ\ <\$>\ n)\ =\ }\<%
\\
\>[0]\AgdaComment{--\ \ \ ∃[\ f\ ]\ ((τ\ ≡t\ (f\ <\$>\ ⇑NE\ n))\ ×\ (SoundKripkeNE\ f\ φ))}\<%
\\
\>[0]\AgdaComment{--\ ⟦\ []\ ⟧r≋\ (zero\ ,\ P)\ =\ ⊤}\<%
\\
\>[0]\AgdaComment{--\ ⟦\ []\ ⟧r≋\ (suc\ n\ ,\ P)\ =\ ⊥}\<%
\\
\>[0]\AgdaComment{--\ ⟦\ x\ ∷\ ρ\ ⟧r≋\ (zero\ ,\ P)\ =\ ⊥}\<%
\\
\>[0]\AgdaComment{--\ ⟦\ x\ ∷\ ρ\ ⟧r≋\ (suc\ n\ ,\ P)\ =\ \ (⟦\ x\ ⟧≋₂\ (P\ fzero))\ ×\ ⟦\ ρ\ ⟧r≋\ (n\ ,\ P\ ∘\ fsuc)}\<%
\\
%
\\[\AgdaEmptyExtraSkip]%
\>[0]\AgdaComment{--\ SoundKripke\ \{Δ₁\ =\ Δ₁\}\ \{κ₁\ =\ κ₁\}\ \{κ₂\ =\ κ₂\}\ f\ F\ =\ \ \ \ \ }\<%
\\
\>[0]\AgdaComment{--\ \ \ \ \ ∀\ \{Δ₂\}\ (ρ\ :\ Renamingₖ\ Δ₁\ Δ₂)\ \{v\ V\}\ →\ }\<%
\\
\>[0]\AgdaComment{--\ \ \ \ \ \ \ ⟦\ v\ ⟧≋\ V\ →\ }\<%
\\
\>[0]\AgdaComment{--\ \ \ \ \ \ \ ⟦\ (renₖ\ ρ\ f\ ·\ v)\ ⟧≋\ (renKripke\ ρ\ F\ ·V\ V)}\<%
\\
%
\\[\AgdaEmptyExtraSkip]%
\>[0]\AgdaComment{--\ SoundKripkeNE\ \{Δ₁\ =\ Δ₁\}\ \{κ₁\ =\ κ₁\}\ \{κ₂\ =\ κ₂\}\ f\ F\ =\ \ \ \ \ }\<%
\\
\>[0]\AgdaComment{--\ \ \ \ \ ∀\ \{Δ₂\}\ (r\ :\ Renamingₖ\ Δ₁\ Δ₂)\ \{v\ V\}\ →\ }\<%
\\
\>[0]\AgdaComment{--\ \ \ \ \ \ \ ⟦\ v\ ⟧≋ne\ \ V\ →\ }\<%
\\
\>[0]\AgdaComment{--\ \ \ \ \ \ \ ⟦\ (renₖ\ r\ f\ ·\ v)\ ⟧≋\ (F\ r\ V)}\<%
\end{code}

\subsubsection{Properties}~
\begin{code}%
\>[0]\AgdaComment{--\ reflect-⟦⟧≋\ :\ ∀\ \{τ\ :\ Type\ Δ\ κ\}\ \{υ\ :\ \ NeutralType\ Δ\ κ\}\ →\ }\<%
\\
\>[0]\AgdaComment{--\ \ \ \ \ \ \ \ \ \ \ \ \ \ τ\ ≡t\ ⇑NE\ υ\ →\ ⟦\ τ\ ⟧≋\ (reflect\ υ)}\<%
\\
\>[0]\AgdaComment{--\ reify-⟦⟧≋\ :\ ∀\ \{τ\ :\ Type\ Δ\ κ\}\ \{V\ :\ \ SemType\ Δ\ κ\}\ →\ }\<%
\\
\>[0]\AgdaComment{--\ \ \ \ \ \ \ \ \ \ \ \ \ \ \ \ ⟦\ τ\ ⟧≋\ V\ →\ τ\ ≡t\ ⇑\ (reify\ V)}\<%
\\
\>[0]\AgdaComment{--\ η-norm-≡t\ :\ ∀\ (τ\ :\ NeutralType\ Δ\ κ)\ →\ ⇑\ (η-norm\ τ)\ ≡t\ ⇑NE\ τ\ }\<%
\\
\>[0]\AgdaComment{--\ subst-⟦⟧≋\ :\ ∀\ \{τ₁\ τ₂\ :\ Type\ Δ\ κ\}\ →\ }\<%
\\
\>[0]\AgdaComment{--\ \ \ τ₁\ ≡t\ τ₂\ →\ \{V\ :\ SemType\ Δ\ κ\}\ →\ ⟦\ τ₁\ ⟧≋\ V\ →\ ⟦\ τ₂\ ⟧≋\ V\ }\<%
\end{code}

\subsubsection{Logical environments}~
\begin{code}%
\>[0]\AgdaComment{--\ ⟦\AgdaUnderscore{}⟧≋e\AgdaUnderscore{}\ :\ ∀\ \{Δ₁\ Δ₂\}\ →\ Substitutionₖ\ Δ₁\ Δ₂\ →\ Env\ Δ₁\ Δ₂\ →\ Set\ \ }\<%
\\
\>[0]\AgdaComment{--\ ⟦\AgdaUnderscore{}⟧≋e\AgdaUnderscore{}\ \{Δ₁\}\ σ\ η\ =\ ∀\ \{κ\}\ (α\ :\ TVar\ Δ₁\ κ)\ →\ ⟦\ (σ\ α)\ ⟧≋\ (η\ α)}\<%
\\
%
\\[\AgdaEmptyExtraSkip]%
\>[0]\AgdaComment{--\ Identity\ relation}\<%
\\
\>[0]\AgdaComment{--\ idSR\ :\ ∀\ \{Δ₁\}\ →\ \ ⟦\ `\ ⟧≋e\ (idEnv\ \{Δ₁\})}\<%
\\
\>[0]\AgdaComment{--\ idSR\ α\ =\ reflect-⟦⟧≋\ eq-refl}\<%
\end{code}
\begin{code}[hide]%
\>[0]\AgdaComment{--\ reflect-⟦⟧≋\ =\ bot\ \AgdaUnderscore{}}\<%
\\
\>[0]\AgdaComment{--\ reify-⟦⟧≋\ =\ bot\ \AgdaUnderscore{}}\<%
\\
\>[0]\AgdaComment{--\ η-norm-≡t\ =\ bot\ \AgdaUnderscore{}}\<%
\\
\>[0]\AgdaComment{--\ subst-⟦⟧≋\ =\ bot\ \AgdaUnderscore{}}\<%
\\
\>[0]\AgdaComment{--\ \textbackslash{}end\{code\}}\<%
\\
\>[0]\AgdaComment{--\ \textbackslash{}subsection\{The\ fundamental\ theorem\ and\ soundness\}}\<%
\\
\>[0]\AgdaComment{--\ \textbackslash{}begin\{code\}}\<%
\\
\>[0]\AgdaComment{--\ fundS\ :\ ∀\ \{Δ₁\ Δ₂\ κ\}(τ\ :\ Type\ Δ₁\ κ)\{σ\ :\ Substitutionₖ\ Δ₁\ Δ₂\}\{η\ :\ Env\ Δ₁\ Δ₂\}\ →\ }\<%
\\
\>[0]\AgdaComment{--\ \ \ \ \ \ \ \ \ \ \ ⟦\ σ\ ⟧≋e\ η\ \ →\ ⟦\ subₖ\ σ\ τ\ ⟧≋\ (eval\ τ\ η)}\<%
\\
\>[0]\AgdaComment{--\ fundSRow\ :\ ∀\ \{Δ₁\ Δ₂\ κ\}(xs\ :\ SimpleRow\ Type\ Δ₁\ R[\ κ\ ])\{σ\ :\ Substitutionₖ\ Δ₁\ Δ₂\}\{η\ :\ Env\ Δ₁\ Δ₂\}\ →\ }\<%
\\
\>[0]\AgdaComment{--\ \ \ \ \ \ \ \ \ \ \ ⟦\ σ\ ⟧≋e\ η\ \ →\ ⟦\ subRowₖ\ σ\ xs\ ⟧r≋\ (evalRow\ xs\ η)}\<%
\\
\>[0]\AgdaComment{--\ fundSPred\ :\ ∀\ \{Δ₁\ κ\}(π\ :\ Pred\ Type\ Δ₁\ R[\ κ\ ])\{σ\ :\ Substitutionₖ\ Δ₁\ Δ₂\}\{η\ :\ Env\ Δ₁\ Δ₂\}\ →\ }\<%
\\
\>[0]\AgdaComment{--\ \ \ \ \ \ \ \ \ \ \ ⟦\ σ\ ⟧≋e\ η\ →\ (subPredₖ\ σ\ π)\ ≡p\ ⇑Pred\ (evalPred\ π\ η)\ }\<%
\end{code}

\begin{code}[hide]%
\>[0]\AgdaComment{--\ fundS\ =\ bot\ \AgdaUnderscore{}}\<%
\\
\>[0]\AgdaComment{--\ fundSRow\ =\ bot\ \AgdaUnderscore{}}\<%
\\
\>[0]\AgdaComment{--\ fundSPred\ =\ bot\ \AgdaUnderscore{}}\<%
\end{code}

\begin{code}%
\>[0]\AgdaComment{--------------------------------------------------------------------------------}\<%
\\
\>[0]\AgdaComment{--\ Fundamental\ theorem\ when\ substitution\ is\ the\ identity}\<%
\\
\>[0]\AgdaComment{--\ subₖ-id\ :\ ∀\ (τ\ :\ Type\ Δ\ κ)\ →\ subₖ\ `\ τ\ ≡\ τ\ }\<%
\\
%
\\[\AgdaEmptyExtraSkip]%
\>[0]\AgdaComment{--\ ⊢⟦\AgdaUnderscore{}⟧≋\ :\ ∀\ (τ\ :\ Type\ Δ\ κ)\ →\ ⟦\ τ\ ⟧≋\ eval\ τ\ idEnv}\<%
\\
\>[0]\AgdaComment{--\ ⊢⟦\ τ\ ⟧≋\ =\ subst-⟦⟧≋\ (inst\ (subₖ-id\ τ))\ (fundS\ τ\ idSR)}\<%
\end{code}
\begin{code}[hide]%
\>[0]\AgdaComment{--\ subₖ-id\ τ\ =\ bot\ \AgdaUnderscore{}}\<%
\end{code}

\begin{code}%
\>[0]\AgdaComment{--------------------------------------------------------------------------------}\<%
\\
\>[0]\AgdaComment{--\ Soundness\ claim\ \ }\<%
\\
%
\\[\AgdaEmptyExtraSkip]%
\>[0]\AgdaComment{--\ soundness\ :\ \ ∀\ \{Δ₁\ κ\}\ →\ (τ\ :\ Type\ Δ₁\ κ)\ →\ τ\ ≡t\ ⇑\ (⇓\ τ)\ }\<%
\\
\>[0]\AgdaComment{--\ soundness\ τ\ =\ reify-⟦⟧≋\ (⊢⟦\ τ\ ⟧≋)}\<%
\\
%
\\[\AgdaEmptyExtraSkip]%
\>[0][@{}l@{\AgdaIndent{0}}]%
\>[1]\AgdaComment{--------------------------------------------------------------------------------}\<%
\\
\>[0]\AgdaComment{--\ If\ τ₁\ normalizes\ to\ ⇓\ τ₂\ then\ the\ embedding\ of\ τ₁\ is\ equivalent\ to\ τ₂}\<%
\\
%
\\[\AgdaEmptyExtraSkip]%
\>[0]\AgdaComment{--\ embed-≡t\ :\ ∀\ \{τ₁\ :\ NormalType\ Δ\ κ\}\ \{τ₂\ :\ Type\ Δ\ κ\}\ \ →\ τ₁\ ≡\ (⇓\ τ₂)\ →\ ⇑\ τ₁\ ≡t\ τ₂}\<%
\\
\>[0]\AgdaComment{--\ embed-≡t\ \{τ₁\ =\ τ₁\}\ \{τ₂\}\ refl\ =\ eq-sym\ (soundness\ τ₂)\ }\<%
\\
%
\\[\AgdaEmptyExtraSkip]%
\>[0]\AgdaComment{--------------------------------------------------------------------------------}\<%
\\
\>[0]\AgdaComment{--\ Soundness\ implies\ the\ converse\ of\ completeness,\ as\ desired}\<%
\\
%
\\[\AgdaEmptyExtraSkip]%
\>[0]\AgdaComment{--\ Completeness⁻¹\ :\ ∀\ \{Δ\ κ\}\ →\ (τ₁\ τ₂\ :\ Type\ Δ\ κ)\ →\ ⇓\ τ₁\ ≡\ ⇓\ τ₂\ →\ τ₁\ ≡t\ τ₂}\<%
\\
\>[0]\AgdaComment{--\ Completeness⁻¹\ τ₁\ τ₂\ eq\ =\ eq-trans\ (soundness\ τ₁)\ (embed-≡t\ eq)}\<%
\end{code}


\bibliographystyle{plainnat}
\bibliography{NBE}
\end{document}
%%% Local Variables: 
%%% TeX-command-extra-options: "-shell-escape"
%%% End:
%  LocalWords:  denotational Agda Wadler dPoint sqrt subtyping coercions Intr
%  LocalWords:  RowTypes Bool eval GHC reified HillerstromL Leijen LindleyM RO
%  LocalWords:  ChapmanKNW Aydemir AbelAHPMSS AbelC AbelOV plfa HubersIMM STLC
%  LocalWords:  MorrisM denotationally DenotationalSoundness RowTheories Suc de
%  LocalWords:  ReifyingVariants RowTheory BerthomieuM CardelliMMS HarperP NatF
%  LocalWords:  XueOX GasterJ Sipser SaffrichTM Env Expr Agda's Leivant ChanW
%  LocalWords:  ThiemannW ImpredicativeSet ImpredicativeSetSucks AbelP chapman
%  LocalWords:  AltenkirchK KaposiKK Gaster XieOBS BiXOS Chlipala objTypes Bahr
%  LocalWords:  Garrigue KEnv PEnv
