\documentclass[dvipsnames,aspectratio=169,pdftex]{beamer}
\usepackage{agda}
\usepackage{stmaryrd}
\usepackage{xcolor}
\usepackage{txfonts}
\usepackage[T1]{fontenc}
\usepackage{microtype}
\DisableLigatures[-]{encoding=T1}
\usepackage{tikz}
\usetikzlibrary{cd}
\usepackage{agda-generated}


\usepackage{dsfont}
\usepackage{newunicodechar}
\newunicodechar{λ}{\ensuremath{\mathnormal\lambda}}
\newunicodechar{σ}{\ensuremath{\mathnormal\sigma}}
\newunicodechar{τ}{\ensuremath{\mathnormal\tau}}
\newunicodechar{π}{\ensuremath{\mathnormal\pi}}
\newunicodechar{ℕ}{\ensuremath{\mathbb{N}}}
\newunicodechar{∷}{\ensuremath{::}}
\newunicodechar{≡}{\ensuremath{\equiv}}
\newunicodechar{≅}{\ensuremath{\cong}}
\newunicodechar{∀}{\ensuremath{\forall}}
\newunicodechar{ᴸ}{\ensuremath{^L}}
\newunicodechar{ᴿ}{\ensuremath{^R}}
\newunicodechar{ʳ}{\ensuremath{^r}}
\newunicodechar{ⱽ}{\ensuremath{^V}}
\newunicodechar{⟧}{\ensuremath{\rrbracket}}
\newunicodechar{⟦}{\ensuremath{\llbracket}}
\newunicodechar{⊤}{\ensuremath{\top}}
\newunicodechar{⊥}{\ensuremath{\bot}}
\newunicodechar{₁}{\ensuremath{_1}}
\newunicodechar{₂}{\ensuremath{_2}}
\newunicodechar{₃}{\ensuremath{_3}}
\newunicodechar{₄}{\ensuremath{_4}}
\newunicodechar{₅}{\ensuremath{_5}}
\newunicodechar{₆}{\ensuremath{_6}}
\newunicodechar{₇}{\ensuremath{_7}}
\newunicodechar{₈}{\ensuremath{_8}}
\newunicodechar{₉}{\ensuremath{_9}}
\newunicodechar{∈}{\ensuremath{\in}}
\newunicodechar{₀}{\ensuremath{_0}}
\newunicodechar{′}{\ensuremath{'}}
\newunicodechar{ˢ}{\ensuremath{^S}}
\newunicodechar{ᴬ}{\ensuremath{^A}}
\newunicodechar{∘}{\ensuremath{\circ}}
\newunicodechar{𝟙}{\ensuremath{\mathds{1}}}  
\newunicodechar{𝟘}{\ensuremath{\mathds{O}}}
% \newunicodechar{𝟙}{\ensuremath{\mathbb{I}}}  
% \newunicodechar{𝟘}{\ensuremath{\mathbb{O}}}
\newunicodechar{ᴾ}{\ensuremath{^P}}
\newunicodechar{ᵀ}{\ensuremath{^T}}
\newunicodechar{⊎}{\ensuremath{\uplus}}
\newunicodechar{ι}{\ensuremath{\iota}}
\newunicodechar{⇐}{\ensuremath{\Leftarrow}}
\newunicodechar{⇒}{\ensuremath{\Rightarrow}}
\newunicodechar{∎}{\ensuremath{\mathnormal\blacksquare}}
\newunicodechar{➙}{\ensuremath{\to^P}}
\newunicodechar{Δ}{\ensuremath{\Delta}}
\newunicodechar{∅}{\ensuremath{\emptyset}}
\newunicodechar{⁺}{\ensuremath{^+}}
\newunicodechar{𝕏}{\ensuremath{\mathbb{X}}}
%\newunicodechar{·}{\ensuremath{\cdot}} %seems to be defined already!
\newunicodechar{∙}{\ensuremath{\bullet}}
\newunicodechar{⁇}{\ensuremath{?}}
\newunicodechar{‼}{\ensuremath{!}}
\newunicodechar{⊕}{\ensuremath{\oplus}}
\newunicodechar{ℤ}{\ensuremath{\mathbb{Z}}}
\newunicodechar{μ}{\ensuremath{\mu}}
\newunicodechar{∃}{\ensuremath{\exists}}
\newunicodechar{⨟}{\ensuremath{\fatsemi}}
\newunicodechar{Σ}{\ensuremath{\Sigma}}
\newunicodechar{ᵣ}{\ensuremath{_r}}
\newunicodechar{ᵢ}{\ensuremath{_i}}
\newunicodechar{≢}{\ensuremath{\nequiv}}
\newunicodechar{≟}{\ensuremath{\stackrel{{\tiny?}}{=}}}
\newunicodechar{≤}{\ensuremath{\le}}
\newunicodechar{ᵇ}{\ensuremath{^b}}
\newunicodechar{𝓣}{\ensuremath{\mathcal{T}}}
\newunicodechar{𝓔}{\ensuremath{\mathcal{E}}}
\newunicodechar{Γ}{\ensuremath{\Gamma}}
\newunicodechar{γ}{\ensuremath{\gamma}}
\newunicodechar{⊔}{\ensuremath{\sqcup}}
\newunicodechar{α}{\ensuremath{\alpha}}
\newunicodechar{η}{\ensuremath{\eta}}
\newunicodechar{ω}{\ensuremath{\omega}}
\newunicodechar{◁}{\ensuremath{\lhd}}
%PT: seems to be already defined
%\newcommand{\lambdabar}{{\mkern0.75mu\mathchar '26\mkern -9.75mu\lambda}}
\newunicodechar{ƛ}{\ensuremath{\lambdabar}}
\newunicodechar{Λ}{\ensuremath{\mathnormal\Lambda}}
\newunicodechar{ρ}{\ensuremath{\rho}}
\newunicodechar{𝓖}{\ensuremath{\mathcal{G}}}
\newunicodechar{ℓ}{\ensuremath{\ell}}
\newunicodechar{♯}{\ensuremath{\sharp}}
\newunicodechar{⇓}{\ensuremath{\Downarrow}}
\newunicodechar{𝓥}{\ensuremath{\mathcal{V}}}
\newunicodechar{∧}{\ensuremath{\wedge}}
\newunicodechar{ₛ}{\ensuremath{_s}}
\newunicodechar{χ}{\ensuremath{\chi}}
\newunicodechar{⊨}{\ensuremath{\models}}
\newunicodechar{⦂}{\ensuremath{\mathbf{:}}}
\newunicodechar{ς}{\ensuremath{\varsigma}}
\newunicodechar{𝓓}{\ensuremath{\mathcal{D}}}
\newunicodechar{∎}{\ensuremath{\square}}
\newunicodechar{■}{\ensuremath{\blacksquare}}
\newunicodechar{↠}{\ensuremath{\twoheadrightarrow}}
\newunicodechar{↪}{\ensuremath{\hookrightarrow}}
\newunicodechar{β}{\ensuremath{\beta}}
\newunicodechar{ξ}{\ensuremath{\xi}}
\newcommand\Aamp{\AgdaFunction{\ensuremath{\&}}}
\newcommand\Att{\AgdaInductiveConstructor{tt}}
\newcommand\Ado{\AgdaKeyword{do}}
\newcommand\AZ{\AgdaDatatype{ℤ}}
\newcommand\Asuc{\AgdaInductiveConstructor{suc}}
\newcommand\Azero{\AgdaInductiveConstructor{zero}}
\newcommand\Anat{\AgdaInductiveConstructor{nat}}
\newcommand\Aint{\AgdaInductiveConstructor{int}}
\newcommand\Abool{\AgdaInductiveConstructor{bool}}
\newcommand\Atend{\AgdaInductiveConstructor{end}}
\newcommand\Atsend[2]{\AgdaInductiveConstructor{‼{\textcolor{black}{\ensuremath{#1}}}∙{\textcolor{black}{\ensuremath{#2}}}}}
\newcommand\Atrecv[2]{\AgdaInductiveConstructor{⁇{\textcolor{black}{\ensuremath{#1}}}∙{\textcolor{black}{\ensuremath{#2}}}}}
\newcommand\Atcfsend[1]{\AgdaInductiveConstructor{‼{\textcolor{black}{\ensuremath{#1}}}}}
\newcommand\Atcfrecv[1]{\AgdaInductiveConstructor{⁇{\textcolor{black}{\ensuremath{#1}}}}}
\newcommand\Atcfcomp[2]{\AgdaInductiveConstructor{{\textcolor{black}{\ensuremath{#1}}}⨟{\textcolor{black}{\ensuremath{#2}}}}}
\newcommand\Atcfskip{\AgdaInductiveConstructor{skip}}
\newcommand\ACSKIP{\AgdaInductiveConstructor{SKIP}}
\newcommand\ACEND{\AgdaInductiveConstructor{END}}
\newcommand\ACCLOSE{\AgdaInductiveConstructor{CLOSE}}
\newcommand\ACfork{\AgdaInductiveConstructor{fork}}
\newcommand\ACconnect{\AgdaInductiveConstructor{connect}}
\newcommand\ACterminate{\AgdaInductiveConstructor{terminate}}
\newcommand\ACdelegateIN{\AgdaInductiveConstructor{delegateIN}}
\newcommand\ACdelegateOUT{\AgdaInductiveConstructor{delegateOUT}}
\newcommand\ACtransmit{\AgdaInductiveConstructor{transmit}}
\newcommand\ACbranch{\AgdaInductiveConstructor{branch}}
\newcommand\ACclose{\AgdaInductiveConstructor{close}}
\newcommand\ACSEND{\AgdaInductiveConstructor{SEND}}
\newcommand\ACRECV{\AgdaInductiveConstructor{RECV}}
\newcommand\ACSELECT{\AgdaInductiveConstructor{SELECT}}
\newcommand\ACCHOICE{\AgdaInductiveConstructor{CHOICE}}
\newcommand\AFin{\AgdaDatatype{Fin}}
\newcommand\AXCommand{\AgdaDatatype{XCmd}}
\newcommand\ACommand{\AgdaDatatype{Cmd}}
\newcommand\ACommandStack{\AgdaDatatype{CmdStack}}
\newcommand\ASession{\AgdaDatatype{Session}}
\newcommand\ASplit{\AgdaDatatype{Split}}
\newcommand\AMSession{\AgdaDatatype{MSession}}
\newcommand\ASet{\AgdaDatatype{Set}}
\newcommand\ASetOne{\AgdaDatatype{Set$_1$}}
\newcommand\ASeto{\AgdaDatatype{Setω}}
\newcommand\Abinaryp{\AgdaFunction{binaryp}}
\newcommand\Aunaryp{\AgdaFunction{unaryp}}
\newcommand\ACheck{\AgdaFunction{Check}}
\newcommand\ACausality{\AgdaFunction{Causality}}
\newcommand\ACheckDual{\AgdaFunction{CheckDual0}}
\newcommand\Aexecutor{\AgdaFunction{exec}}
\newcommand\Aexec{\AgdaFunction{exec}}
\newcommand\AIO{\AgdaFunction{IO}}
\newcommand\Amu{\AgdaInductiveConstructor{\ensuremath{\mu}}}
\newcommand\AMU{\AgdaInductiveConstructor{LOOP}}
\newcommand\AUNROLL{\AgdaInductiveConstructor{UNROLL}}
\newcommand\ACONTINUE{\AgdaInductiveConstructor{CONTINUE}}
\newcommand\Aback{\AgdaInductiveConstructor{\ensuremath{`}}}
\newcommand\Amanyunaryp{\AgdaFunction{many-unaryp}}
\newcommand\Arestart{\AgdaFunction{restart}}
\newcommand\ASerialize{\AgdaRecord{Serialize}}
\newcommand\ARawMonad{\AgdaRecord{RawMonad}}
\newcommand\AReaderT{\AgdaRecord{ReaderT}}
\newcommand\AStateT{\AgdaRecord{StateT}}
\newcommand\Aput{\AgdaFunction{put}}
\newcommand\Amodify{\AgdaFunction{modify}}
\newcommand\Aget{\AgdaFunction{get}}
\newcommand\Aproject{\AgdaFunction{project}}
\newcommand\AlocateSplit{\AgdaFunction{locate-split}}
\newcommand\Aadjust{\AgdaFunction{adjust}}
\newcommand\Aid{\AgdaFunction{id}}
\newcommand\AprimSend{\AgdaFunction{primSend}}
\newcommand\AprimRecv{\AgdaFunction{primRecv}}
\newcommand\Aleafp{\AgdaFunction{leafp}}
\newcommand\Abranchp{\AgdaFunction{branchp}}
\newcommand\Atreep{\AgdaFunction{treep}}
\newcommand\AIntTree{\AgdaDatatype{IntTree}}
\newcommand\AIntTreeF{\AgdaFunction{IntTreeF}}
\newcommand\ACLeaf{\AgdaInductiveConstructor{Leaf}}
\newcommand\ACBranch{\AgdaInductiveConstructor{Branch}}
\newcommand\AtoN{\AgdaFunction{toℕ}}
\newcommand\Asplit{\AgdaFunction{split}}
\newcommand\Across{\AgdaFunction{cross}}
\newcommand\Ajoin{\AgdaFunction{join}}
\newcommand\AsplitTree{\AgdaFunction{splitTree}}
\newcommand\Acont{\AgdaFunction{cont}}
\newcommand\Adollar{\AgdaOperator{\AgdaFunction{\AgdaUnderscore{}\$\AgdaUnderscore{}}}}
\newcommand\AESem{\AgdaOperator{\AgdaFunction{𝓔⟦\AgdaUnderscore{}⟧}}}
\newcommand\AGSem{\AgdaOperator{\AgdaDatatype{𝓖⟦\AgdaUnderscore{}⟧}}}
\newcommand\ATSem{\AgdaOperator{\AgdaDatatype{𝓣⟦\AgdaUnderscore{}⟧}}}
\newcommand\ADEnv{\AgdaDatatype{DEnv}}

%%% Local Variables:
%%% mode: latex
%%% TeX-master: "main-tyde23"
%%% End:


\usetheme{Madrid}

\title{Relating System F Semantics in Agda}
\author[Saffrich, Thiemann, Weidner] {
  Hannes Saffrich \and 
  Peter Thiemann \and
  Marius Weidner
}
\institute{University of Freiburg}
\date{April 28, 2024 (40. GI Workshop, Bad Honnef)}

\newcommand{\SubItem}[1]{
    {\setlength\itemindent{15pt} \item[-] #1}
}

\AtBeginSection[]{%
  \begin{frame}<beamer>
    \frametitle{Outline}
    \tableofcontents[currentsection]%[sectionstyle=show/show,subsectionstyle=hide/show/hide]
  \end{frame}
  \addtocounter{framenumber}{-1}% If you don't want them to affect the slide number
}

\newenvironment{AgdaBlock}{
  \vspace{\AgdaEmptySkip}
  \AgdaNoSpaceAroundCode{}
}{
  \AgdaSpaceAroundCode{}
}

\begin{document}
\begin{frame}{\null}
  \titlepage 
\end{frame}

\begin{frame}[fragile]
  \frametitle{Overview}
  \begin{itemize}
    \item What is finitely stratified System F $SF_2$ \cite{DBLP:journals/iandc/Leivant91}
    \item $SF_2$ is amenable to intrinsically typed syntax
    \item Three semantics for $SF_2$: small-step, big-step \& denotational
    \item .. connected via logical relation
    \item .. in Agda
  \end{itemize}
\end{frame}

\begin{frame}
  \frametitle{Intrinsically Typed Syntax for $SF_2$}
  \framesubtitle{Types}
  \TFLEnv
  \vspace{-5mm}
  \TFType
  \vspace{-5mm}
  \begin{itemize}
    \item Polymorphic lambda calculus \cite{girard72:_inter,DBLP:conf/programm/Reynolds74}
    \item Each type has a level
    \item Quantification only possible over types at lower level
    \item Predicativity is retained
    \item \dots{} enables a set-based semantics
  \end{itemize}
\end{frame}

\begin{frame}
  \frametitle{Intrinsically Typed Syntax for $SF_2$}
  \framesubtitle{Type Contexts}
  \TFTVEnv
  \TFinn
  \begin{itemize}
    \item Single environment for type and expression variables inspired by \cite{DBLP:conf/mpc/ChapmanKNW19}
  \end{itemize}
\end{frame}

\begin{frame}
  \frametitle{Intrinsically Typed Syntax for $SF_2$}
  \framesubtitle{Expressions}
  \TFExprExcerpt
\end{frame}

\begin{frame}
  \frametitle{Operational Semantics of $SF_2$}
  \framesubtitle{Small \& Big Step Semantics}
  \SingleReductionExcerpt
  \SemanticsExcerpt
\end{frame}

\begin{frame}
  \frametitle{Denotational Semantics of $SF_2$}
  \framesubtitle{Types}
  \TFTEnv
  \TFTSem
  \begin{itemize}
    \item Leivant’s levels correspond to Agda’s universe levels
    \item .. and thus \AgdaFunction{Env*} needs to live in \AgdaFunction{Setω}!
  \end{itemize}
\end{frame}

\begin{frame}
  \frametitle{Problem \#1: $Set \omega$ Equality}
  \NormalEqDef
  \vspace{-7.5mm} 
  \OmegaEqDef
  \vspace{-7.5mm} 
  \begin{itemize}
    \item \AgdaFunction{Setω} is Agda’s sort that contains \AgdaFunction{Set} $ℓ$, for all $ℓ$
    \item .. but equality in Agda is only defined for terms of type \AgdaFunction{Set} $ℓ$, for all $ℓ$.
    \item .. Thus we need copies for all lemmas that include reasoning about \AgdaFunction{Setω} equality
    \item Even worse, we need the same for heterogeneous equality
  \end{itemize}
  \begin{exampleblock}{Proposal}
    Extend the level hierarchy in Agda to include a larger subset (e.g. $\epsilon_0$) of ordinals than $\omega$.
  \end{exampleblock}
\end{frame}

\begin{frame}
  \frametitle{Denotational Semantics of $SF_2$}
  \framesubtitle{Expressions}
  \TFVEnv
  \TFExprSem
\end{frame}

\begin{frame}[fragile]
  \frametitle{Relating Operational and Denotational Semantics}
  \framesubtitle{Birds eye view on the theorems}
  \begin{tikzcd}
    small step \arrow[rr, "simulates", dashed] \arrow[rdd, bend left] &                                                                  & big step \arrow[ldd, "soundness \ "', bend right] \\
                                                                      &                                                                  &                                                   \\
                                                                      & denotational \arrow[ruu, "adequacy \ (using \ LR)"', bend right] &                                                  
    \end{tikzcd}
    \vspace{5mm}
    \begin{itemize}
      \item Simulation is easy: the reflexive transitive closure of the small-step relation corresponds to big-step semantics
      \item Soundness can be proven by induction
      \item .. but adequacy requires a binary logical relation 
    \end{itemize}
\end{frame}

\begin{frame}[fragile]
  \frametitle{Soundness}
  \BigStepSoundnessType
  \SmallStepSoundness
  \vspace{-12.5mm} 
  \SmallStepSoundnessProofExcerpt
\end{frame}

\begin{frame}[fragile]
  \frametitle{Problem \#2: Subst Hell}
  \SubDefESub
  \vspace{-7.5mm} 
  \SubstExamplesFusionESubESub
  \vspace{-7.5mm} 
  \begin{itemize}
    \item When \AgdaFunction{subst} appears in the \emph{type} of a proof, things get complicated
    \item There exists a somewhat `mechanical' but tedious way to prove these lemmas involving heterogeneous equality
  \end{itemize}
  \begin{exampleblock}{Proposal}
    Build a solver for equality reasoning over \AgdaFunction{subst} terms using heterogeneous equality. 
  \end{exampleblock}
\end{frame}

\begin{frame}[fragile]
  \frametitle{Adequacy}
  \FundamentalAdequacyType
  \vspace{-12.5mm} 
  \FundamentalAdequacyBody
  \begin{itemize}
    \item The \AgdaFunction{adequacy} theorem follows directly from
      the \AgdaFunction{fundamental} theorem of the logical relation
      % that itself is mostly based on \cite{DBLP:journals/corr/abs-1907-11133,ahmed23:_oplss}
    \item .. and also suffers from `subst hell'
  \end{itemize}
\end{frame}


\begin{frame}[fragile]
  \frametitle{Conclusion}
  \framesubtitle{}
  \begin{itemize}
    \item We mechanized the following theorems in Agda: 
  \end{itemize}
  \begin{tikzcd}
    small step \arrow[rr, "simulates", dashed] \arrow[rdd, bend left] &                                                                  & big step \arrow[ldd, "soundness \ "', bend right] \\
                                                                      &                                                                  &                                                   \\
                                                                      & denotational \arrow[ruu, "adequacy \ (using \ LR)"', bend right] &                                                  
  \end{tikzcd}
  \begin{itemize} 
    \item On paper those theorems are well studied
    \item .. but mechanizing them opens up technical challenges:
    \begin{itemize} 
      \item Soundness and adequacy (i.e. the logical relation) theorems require fusion lemmas for type substitution indexed expression substitutions, leading to `subst hell'
      \item Denotational semantics for languages with level quantification make use of \AgdaFunction{Setω} and require reasoning about propositional- \& homogeneous `$\omega$ equality'
    \end{itemize}
  \end{itemize}
  \begin{verbatim} 
  >>> find src/ -name '*.agda' -print0 | xargs -0 cat | wc -l
  10889
  \end{verbatim}
\end{frame}

\begin{frame}[fragile]
  \frametitle{Sources}  
  \bibliographystyle{ACM-Reference-Format} 
  \bibliography{references}
\end{frame}

\end{document}
